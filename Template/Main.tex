% File: C:/Kochbuch/Main.latex
\documentclass[11pt,oneside]{book}

\usepackage[T1]{fontenc}
\usepackage[utf8]{inputenc}
\usepackage[english]{babel}
\usepackage{lmodern}
\usepackage[a4paper,margin=2.5cm]{geometry}
\usepackage{graphicx}
\usepackage{enumitem}
\usepackage{xcolor}
\usepackage{hyperref}
\usepackage{parskip}
\usepackage{booktabs}

\hypersetup{
    colorlinks=true,
    linkcolor=blue,
    urlcolor=blue,
    pdftitle={Kochbuch},
    pdfauthor={Bernd Mattern}
}

% Simple recipe environment
\newcommand{\RecipeMeta}[2]{\noindent\textbf{Für} #1 {Personen}\quad | \quad \textbf{Zubereitungsdauer:} #2\vspace{4pt}\par}
\newenvironment{ingredients}{\paragraph{Zutaten:}\begin{itemize}[leftmargin=*]}{\end{itemize}}
\newenvironment{directions}{\paragraph{Anleitung:}\begin{enumerate}[leftmargin=*]}{\end{enumerate}}
\newcommand{\Notes}[1]{\paragraph{Notes:}#1}

\title{Mein Kochbuch}
\author{Bernd Mattern}
\date{\today}

\begin{document}
\maketitle
\frontmatter
\renewcommand{\contentsname}{Inhalt}
\tableofcontents
\mainmatter

\chapter{Frühstück}
\section{Classic Pancakes}
\RecipeMeta{4}{20 min}
\begin{ingredients}
    \item 200 g plain flour
    \item 2 tbsp sugar
    \item 2 tsp baking powder
    \item pinch of salt
    \item 300 ml milk
    \item 1 large egg
    \item 2 tbsp melted butter (plus extra for frying)
\end{ingredients}
\begin{directions}
    \item Mix dry ingredients in a bowl.
    \item Whisk milk, egg and melted butter; combine with dry mixture until just blended.
    \item Heat a non-stick pan over medium heat, add a little butter.
    \item Pour 2–3 tablespoons batter per pancake; cook until bubbles form, flip and cook 1–2 minutes more.
    \item Serve warm with syrup, fruit or yogurt.
\end{directions}
\Notes{Do not overmix the batter; small lumps are fine. For fluffier pancakes let batter rest 10 minutes.}

\chapter{Salads \& Starters}
\section{Caesar Salad (Simple)}
\RecipeMeta{2--3}{15 min}
\begin{ingredients}
    \item 1 romaine lettuce, washed and chopped
    \item 50 g grated Parmesan
    \item 2 slices bread, cubed (for croutons)
    \item 2 tbsp olive oil
    \item Dressing: 1 egg yolk (or 1 tbsp mayonnaise), 1 tsp mustard, 1 small garlic clove (minced), 2 tbsp lemon juice, 3 tbsp olive oil, salt and pepper
\end{ingredients}
\begin{directions}
    \item Make croutons: toss bread cubes with 2 tbsp olive oil, toast in oven at 200°C until golden.
    \item Whisk dressing ingredients together until emulsified.
    \item Toss lettuce with dressing, top with croutons and Parmesan.
\end{directions}
\Notes{Anchovies are traditional in Caesar dressing — add 2 chopped anchovy fillets if desired.}

\chapter{Main Courses}
\section{Spaghetti Bolognese}
\RecipeMeta{4}{1 h}
\begin{ingredients}
    \item 350 g spaghetti
    \item 400 g minced beef (or mixed pork/beef)
    \item 1 onion, finely chopped
    \item 2 cloves garlic, minced
    \item 1 carrot, finely diced
    \item 1 celery stalk, finely diced
    \item 400 g canned tomatoes (crushed)
    \item 2 tbsp tomato paste
    \item 150 ml beef or vegetable stock
    \item 1 bay leaf, salt, pepper, olive oil
\end{ingredients}
\begin{directions}
    \item Heat oil in a pan; sauté onion, carrot and celery until soft.
    \item Add garlic and mince; brown the meat thoroughly.
    \item Stir in tomato paste, canned tomatoes and stock; add bay leaf.
    \item Simmer gently 30–40 minutes until thick; season to taste.
    \item Cook spaghetti according to package, drain and combine with sauce or serve sauce on top.
\end{directions}
\Notes{For richer flavor, simmer longer or finish with a splash of milk or cream.}

\section{Vegetable Stir-Fry (Quick)}
\RecipeMeta{2--3}{20 min}
\begin{ingredients}
    \item 1 red pepper, sliced
    \item 1 small broccoli (florets)
    \item 1 carrot, julienned
    \item 150 g snap peas
    \item 2 tbsp soy sauce, 1 tbsp sesame oil, 1 tsp honey or sugar, 1 garlic clove minced, 1 tsp grated ginger
    \item 2 tbsp vegetable oil for frying
\end{ingredients}
\begin{directions}
    \item Mix soy sauce, sesame oil, honey, garlic and ginger into a sauce.
    \item Heat wok or large pan on high, add oil and stir-fry harder vegetables first (carrot, broccoli) 2–3 minutes.
    \item Add remaining vegetables, pour sauce, toss 1–2 minutes until vegetables are crisp-tender.
    \item Serve over rice or noodles.
\end{directions}

\chapter{Desserts}
\section{Simple Chocolate Cake}
\RecipeMeta{8}{45 min (including baking)}
\begin{ingredients}
    \item 200 g sugar
    \item 175 g plain flour
    \item 75 g cocoa powder
    \item 1.5 tsp baking powder
    \item 1.5 tsp bicarbonate of soda
    \item 2 eggs
    \item 240 ml milk
    \item 120 ml vegetable oil
    \item 2 tsp vanilla extract
\end{ingredients}
\begin{directions}
    \item Preheat oven to 180°C. Grease and line a 23 cm cake tin.
    \item Mix dry ingredients in a bowl. Whisk eggs, milk, oil and vanilla; combine with dry mix until smooth.
    \item Pour into tin and bake 30–35 minutes or until a skewer comes out clean.
    \item Cool on a rack before slicing; optionally top with frosting or dust with icing sugar.
\end{directions}
\Notes{This recipe is easily halved for a smaller cake. Use espresso in place of some milk for deeper chocolate flavor.}

\backmatter
\chapter*{Index}
\addcontentsline{toc}{chapter}{Index}
% Simple manual index entries (expand as needed)
Pancakes \dotfill 1\\
Spaghetti Bolognese \dotfill 6\\
Chocolate Cake \dotfill 10

\end{document}