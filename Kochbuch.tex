\documentclass[11pt,oneside]{book}

\usepackage[T1]{fontenc}
\usepackage[utf8]{inputenc}
\usepackage[english]{babel}
\usepackage{lmodern}
\usepackage[a4paper,margin=2.5cm]{geometry}
\usepackage{graphicx}
\usepackage{enumitem}
\usepackage{xcolor}
\usepackage{parskip}
\usepackage{booktabs}
\usepackage[colorlinks]{hyperref}
\usepackage{imakeidx}
\usepackage{tocloft}
\graphicspath{{Quellen/}}

\makeindex[title=Index, intoc]  % Index in ToC aufnehmen

% Simple recipe environment
\newcommand{\RecipeMeta}[2]{\noindent\textbf{Für} #1 {Personen}\quad | \quad \textbf{Zubereitungsdauer:} #2 {min}\vspace{4pt}\par}
\newenvironment{ingredients}{\paragraph{Zutaten:}\begin{itemize}[leftmargin=*]}{\end{itemize}}
\newenvironment{directions}{\paragraph{Anleitung:}\begin{enumerate}[leftmargin=*]}{\end{enumerate}}
\newcommand{\Notes}[1]{\paragraph{Notizen:}#1}

\title{Gesammelte Rezepte}
\author{von Bernd Mattern}

\begin{document}
\maketitle
\frontmatter
\renewcommand{\contentsname}{Inhalt}
\renewcommand{\chaptername}{Kapitel}
\renewcommand{\listfigurename}{Bilderverzeichnis}
\cftsetindents{section}{1.5em}{3em}
\tableofcontents 
\mainmatter

\chapter{Grundrezepte}
\section{Hefeteig (süß oder salzig)}
\index{Hefeteig (süß oder salzig)}
\RecipeMeta{8}{90}
\begin{ingredients}
  \item 25 g frische Hefe (oder 1 Päckchen Trockenhefe)
  \item 300 ml lauwarme Milch
  \item 1 Prise Zucker
  \item 500 g Mehl
  \item 40 g Zucker (für süße Variante, weglassen bei salziger)
  \item 1 TL Salz (für salzige Variante etwas mehr)
  \item 60 g Sanella oder Butter, zerlassen
\end{ingredients}
\begin{directions}
  \item Hefe in lauwarmer Milch mit 1 Prise Zucker auflösen und 5-10 Minuten an einem warmen Ort zugedeckt gehen lassen.
  \item Mehl, Zucker (bei süßer Variante), Salz und zerlassenes Fett in eine Schüssel geben.
  \item Die Hefemilch dazugeben und alles kräftig zu einem glatten Teig kneten.
  \item Teig mit Folie abdecken und an einem warmen Ort gehen lassen, bis er sich verdoppelt hat.
  \item Dann nochmals durchkneten und nach Belieben weiterverarbeiten (z. B. zu Hefezopf, Kleingebäck, Weißbrot, Butter- oder Obstkuchen).
  \item Für salziges Gebäck den Zucker weglassen.
  \item Bei Verwendung von Trockenhefe: Inhalt des Päckchens mit 1 TL Zucker und 5 EL lauwarmer Milch (von der Gesamtmenge) verrühren, 15 Minuten gehen lassen und dann wie oben fortfahren.
\end{directions}
\Notes{Nicht über 60°C erhitzen - Hefe stirbt sonst ab. Teig kräftig kneten, ggf. Küchenmaschine nutzen. Lässt sich 2-3 Monate tiefgefrieren; dafür halbe Hefemenge und etwas mehr Zucker verwenden. Erst nach dem Auftauen gehen lassen.}
\newpage
\section{Streusel}
\index{Streusel}
\RecipeMeta{1}{10}
\begin{ingredients}
  \item 1 Teil Butter
  \item 1 Teil Zucker
  \item 2 Teile Mehl
  \item 
  \item 1 Teil = 250g für eine großes Blech
\end{ingredients}
\begin{directions}
  \item Butter erwärmen, aber nicht schmelzen.
  \item Alles gleichmäßig miteinander verkneten.
  \item Aus dem Teig mit der Hand teelöffelgroße Flocken formen und
  \item auf dem Kuchen verweilen.
\end{directions}
\begin{figure}[h]
\centering
\includegraphics[width=0.75\textwidth]{Bilder/Streusel.jpg}
\caption{Streusel}
\end{figure}
\newpage
\section{Grundrezept Quarkblätterteig}
\index{Grundrezept Quarkblätterteig}
\RecipeMeta{1}{60}
\begin{ingredients}
  \item 250 g Magerquark
  \item 250 g Mehl
  \item 1 Msp. Backpulver
  \item Salz
  \item 250 g Sanella, gut gekühlt
\end{ingredients}
\begin{directions}
  \item Quark in einem Tuch gut ausdrücken.
  \item Mehl auf ein Backbrett geben, Backpulver und Salz darüberstreuen.
  \item Quark und Sanella daraufgeben und alles gut mit dem Pfannenmesser durchhacken.
  \item Schnell zu einem Teig verkneten.
  \item Ca. 1 Stunde im Kühlschrank ruhen lassen.
  \item Dann den Teig wie echten Blätterteig in verschiedenen Touren weiter verarbeiten.
\end{directions}
\Notes{Kann anstelle von Blätterteig für alle Blätterteigrezepte verwendet werden. Haltbarkeit: zugedeckt im Kühlschrank 2–3 Tage, tiefgefroren 2–3 Monate.}
\newpage
\section{Mürbteig oder Knetteig}
\index{Mürbteig oder Knetteig}
\RecipeMeta{1}{10}
\begin{ingredients}
  \item Mehl
  \item Zucker
  \item Mandeln oder Nüsse
  \item Ei oder Zitronensaft
  \item Butter oder Margarine
\end{ingredients}
\begin{directions}
  \item Mehl sieben, Zucker und Nüsse mischen.
  \item Ei oder Zitronensaft in Mulde geben.
  \item Butter zugeben, alles verkneten.
  \item Kalt stellen.
\end{directions}
\Notes{Grundteig für viele Backwaren.}
\newpage
\section{Grundrezept Brandteig}
\index{Grundrezept Brandteig}
\RecipeMeta{1}{20}
\begin{ingredients}
  \item 1/4 l Wasser
  \item 1 Prise Salz
  \item 50 g Sanella
  \item 150 g Mehl
  \item 1 Ei
  \item 3–4 Eier
  \item 1 TL Backpulver
\end{ingredients}
\begin{directions}
  \item Wasser, Salz, Sanella aufkochen.
  \item Mehl auf einmal zugeben, abbrennen.
  \item Ein Ei unterrühren, abkühlen lassen.
  \item Restliche Eier und Backpulver unterrühren.
\end{directions}
\Notes{Für Windbeutel, Spritzkuchen geeignet.}
\newpage
\section{Grundrezept Blätterteig}
\index{Grundrezept Blätterteig}
\RecipeMeta{1}{30}
\begin{ingredients}
  \item 75 g Mehl
  \item 200 g Sanella (gut gekühlt)
  \item 175 g Mehl
  \item 50 g Sanella
  \item 100 g Wasser (sehr kalt)
  \item 1 Prise Salz
\end{ingredients}
\begin{directions}
  \item 75 g Mehl mit 200 g Sanella verkneten, zu Rechteck ausrollen, kalt stellen.
  \item 175 g Mehl, 50 g Sanella, Salz und Wasser zu Teig verkneten.
  \item Teig ausrollen, Fettplatte in die Mitte legen, einschlagen.
  \item Teig ausrollen, falten, kalt stellen.
  \item Touren 4-mal wiederholen, jeweils 15 Min. kühlen.
  \item Teig für süße und salzige Backwaren verwenden.
\end{directions}
\Notes{Haltbarkeit: zugedeckt im Kühlschrank 2–3 Tage, tiefgefroren 2–3 Monate.}
\newpage
\section{Hefeteig}
\index{Hefeteig}
\RecipeMeta{1}{0}
\begin{ingredients}
  \item 
\end{ingredients}
\begin{directions}
  \item 
\end{directions}
\begin{figure}[h]
\centering
\includegraphics[width=0.75\textwidth]{Bilder/Hefeteig.jpg}
\caption{Hefeteig}
\end{figure}
\newpage
\section{Hefeteig (Grundrezept Nr. 1241)}
\index{Hefeteig (Grundrezept Nr. 1241)}
\RecipeMeta{1}{None}
\begin{ingredients}
  \item 500 g Mehl (Type 405)
  \item 125–150 g Butter oder Margarine
  \item 30–40 g Hefe
  \item 1 Ei
  \item 1/8 l Milch
  \item 1 Prise Salz
  \item Optional: Zucker je nach Rezept
\end{ingredients}
\begin{directions}
  \item Mehl in Schüssel sieben, Salz zugeben.
  \item Hefe in lauwarmer Milch auflösen, mit Ei und Butter verrühren.
  \item Flüssigkeit zum Mehl geben, zu glattem Teig verarbeiten.
  \item Teig zugedeckt an warmem Ort gehen lassen, bis er sich verdoppelt hat.
  \item Teig ausrollen und nach Rezept weiterverarbeiten.
\end{directions}
\Notes{Handschriftliche Notiz: Für Rohrnudeln: 100 g Zucker, 2 Eier, 3 Eigelb, 300–350 ml Milch, 100 g Fett.}
\newpage
\section{Nudelteig (Grundrezept Nr. 917)}
\index{Nudelteig (Grundrezept Nr. 917)}
\RecipeMeta{1}{None}
\begin{ingredients}
  \item 250–300 g Mehl
  \item Salz
  \item 2 Eier
  \item 2–4 EL Wasser (je nach Mehlmenge und Eiergröße)
  \item Optional: 1 EL Öl (Handschriftliche Notiz)
\end{ingredients}
\begin{directions}
  \item Mehl auf Brett sieben, Salz zugeben.
  \item In der Mehlgrube Wasser und Eier nach und nach vorsichtig einrühren.
  \item Teig herstellen, restliches Mehl einarbeiten, bis Teig glatt ist.
  \item Teig in 2–3 Portionen teilen, ausrollen und trocknen lassen.
  \item Für Nudeln: Streifen schneiden, auf Brett trocknen.
  \item Bei elektrischen Geräten: Mehl und Flüssigkeit in Schüssel geben, mit Knethaken verarbeiten.
\end{directions}
\Notes{Handschriftliche Notiz: 'Nach Oma: 1 Ei, 1/8 l Milch statt Wasser, 1 EL Öl'. Verwendung: Suppennudeln, Maultaschen, Ravioli, Cannelloni, Süßspeisen.}
\newpage
\chapter{Rohkost und Salate}
\section{Paprikasalat mit Salami}
\index{Paprikasalat mit Salami}
\RecipeMeta{4}{15}
\begin{ingredients}
  \item 4 Paprikaschoten
  \item 1 große Zwiebel
  \item 100 g ungarische Salami
  \item 1 Bund Petersilie
  \item 3 EL Essig
  \item 2 EL Öl
  \item 2 Knoblauchzehen
  \item Salz
  \item Pfeffer
\end{ingredients}
\begin{directions}
  \item Paprikaschoten halbieren, entkernen, waschen, abtrocknen und in dünne Streifen schneiden.
  \item Zwiebel schälen und in feine Ringe schneiden. Salami in dünne Scheiben schneiden und vierteln.
  \item Alles in einer Schüssel mischen. Petersilie waschen, trockentupfen, fein hacken und darüberstreuen.
  \item Salat mit Essig, Öl, Pfeffer würzen. Knoblauch schälen, klein schneiden, mit Salz bestreuen, zerdrücken und über den Salat geben.
  \item Salat gut durchmischen und 8 Minuten ziehen lassen.
\end{directions}
\Notes{Frisch servieren, passt gut zu Brot oder als Beilage.}
\newpage
\section{Reissalat}
\index{Reissalat}
\RecipeMeta{1}{30}
\begin{ingredients}
  \item 1 Tasse Reis
  \item 1 große Paprika, gewürfelt
  \item 3 Tomaten, gewürfelt
  \item 1 Zwiebel, gehackt
  \item Gauda, gestiftet
  \item Fleischwurst, gewürfelt
  \item Ananas (Stückchen)
  \item Curry
  \item Brühwürfel
  \item Für die Soße:
  \item Ananassaft (aus der Dose)
  \item Senf
  \item Worchester-Sauce
  \item Essig
  \item Zwiebeln
  \item Maderia
  \item Curry
  \item Paprika
  \item Öl
  \item Saure Sahne
\end{ingredients}
\begin{directions}
  \item 1 Tasse Reis mit Brühwürfel kochen, abtropfenlassen, nochwarm mit Curry mischen
  \item Ziehen lassen.
  \item Aus Ananassaft, Senf, Worchester-Sauce, Essig, Zwiebeln, Maderia, Curry, Paprika, Öl, Saure Sahne die Sauce herstellen.
\end{directions}
\begin{figure}[h]
\centering
\includegraphics[width=0.75\textwidth]{Bilder/Reissalat.jpg}
\caption{Reissalat}
\end{figure}
\Notes{1 Kaffeetopf als Menge}
\newpage
\section{Fenchelrohkost}
\index{Fenchelrohkost}
\RecipeMeta{1}{15}
\begin{ingredients}
  \item 1 Fenchelknolle
  \item 2 EL French Dressing
  \item Pfeffer
\end{ingredients}
\begin{directions}
  \item Fenchelknolle putzen, waschen, halbieren und ganz fein schneiden.
  \item Mit French Dressing und Pfeffer mischen.
  \item Etwa 15 Minuten durchziehen lassen.
\end{directions}
\Notes{Einfach und frisch.}
\newpage
\section{Bunter Fenchelsalat}
\index{Bunter Fenchelsalat}
\RecipeMeta{3}{20}
\begin{ingredients}
  \item 4 Fenchelknollen
  \item 1 grüne Paprikaschote
  \item 1 Zwiebel
  \item 2 Tomaten
  \item Salz, Pfeffer
  \item 1 TL Senf
  \item 1 Prise Zucker
  \item 3 EL Öl
  \item 1 EL Weinbrand
  \item 200 g gekochter Schinken
  \item Basilikum
  \item Petersilie
\end{ingredients}
\begin{directions}
  \item Fenchel putzen, waschen und in dünne Ringe schneiden.
  \item Paprika halbieren, entkernen und in Streifen schneiden.
  \item Zwiebel würfeln, Tomaten brühen, abziehen und achteln.
  \item Alles mischen und Marinade aus Salz, Pfeffer, Senf, Zucker, Öl und Weinbrand zubereiten.
  \item Über die Zutaten gießen, 10 Minuten ziehen lassen.
  \item Schinken und Kräuter untermischen.
\end{directions}
\Notes{Mit Schinken und Kräutern verfeinert.}
\newpage
\chapter{Brotaufstrich}
\section{Petersilienpesto}
\index{Petersilienpesto}
\RecipeMeta{4}{15}
\begin{ingredients}
  \item 1 Bund frische Petersilie
  \item 2 Knoblauchzehen
  \item 2–3 EL Weißweinessig oder Zitronensaft
  \item 80–100 ml Olivenöl
  \item 2–3 EL Mandeln (alternativ Pinienkerne oder Walnüsse)
  \item 30 g frisch geriebener Parmesan
  \item Salz
  \item Pfeffer
\end{ingredients}
\begin{directions}
  \item Petersilie waschen, trocken schütteln und grob hacken.
  \item Knoblauch schälen und grob zerkleinern.
  \item Mandeln in einer Pfanne ohne Fett leicht anrösten (optional) und abkühlen lassen.
  \item Petersilie, Knoblauch, Mandeln, Parmesan, Essig, Salz und Pfeffer in einen Mixer oder Mörser geben.
  \item Nach und nach das Olivenöl zugeben und alles zu einer glatten, cremigen Paste verarbeiten.
  \item Mit Salz, Pfeffer und ggf. etwas mehr Essig abschmecken.
  \item In ein sauberes Glas füllen und mit einer dünnen Schicht Olivenöl bedecken, um das Pesto länger haltbar zu machen.
\end{directions}
\Notes{Passt hervorragend zu Pasta, Kartoffeln, gegrilltem Gemüse oder als Brotaufstrich. Im Kühlschrank mit Öl bedeckt ca. 1 Woche haltbar.}
\newpage
\section{Dattelcreme}
\index{Dattelcreme}
\RecipeMeta{10}{10}
\begin{ingredients}
  \item 90g Datteln
  \item 235g Frischkäse
  \item 250g Schmand
  \item 1 TL Currypulver
  \item 1/2 TL Salz
  \item 1 Messerspitze Sambal Olek
\end{ingredients}
\begin{directions}
  \item Datteln in einem Mixer zerkleinern und 
  \item mit den restlichen Zutaten vermischen
\end{directions}
\begin{figure}[h]
\centering
\includegraphics[width=0.75\textwidth]{Bilder/Dattelcreme.jpg}
\caption{Dattelcreme}
\end{figure}
\Notes{von Felix G.}
\newpage
\chapter{Schnelle Küche}
\section{Kräutertoast àl Ida}
\index{Kräutertoast àl Ida}
\RecipeMeta{2}{20}
\begin{ingredients}
  \item 4 Toastscheiben
  \item 2 Ecken Kräuterschmelzkäse (je ca. 60 g)
  \item 50 g Butter, geschmolzen
  \item 150 g gekochter Schinken, klein geschnitten
  \item Schnittlauch, fein gehackt
\end{ingredients}
\begin{directions}
  \item Den Kräuterschmelzkäse mit der geschmolzenen Butter verrühren.
  \item Den klein geschnittenen Schinken und den gehackten Schnittlauch untermischen.
  \item Die Mischung gleichmäßig auf die Toastscheiben streichen.
  \item Die Toasts bei 250°C etwa 10-15 Minuten überbacken, bis sie goldbraun sind.
\end{directions}
\Notes{Schmeckt besonders gut frisch aus dem Ofen. Optional mit etwas Paprikapulver oder Pfeffer würzen.}
\newpage
\section{Pizza Toast}
\index{Pizza Toast}
\RecipeMeta{1}{20}
\begin{ingredients}
  \item Vollkorntoast
  \item gekochter Schinken
  \item Parmigiano
  \item Oregano
  \item Thymian
  \item Pfeffer
  \item Gauda
\end{ingredients}
\begin{directions}
  \item Zutaten vermischen
  \item Toast mit Mischung bestreichen
  \item 7-8min bei 200°C O/U Hitze backen
\end{directions}
\begin{figure}[h]
\centering
\includegraphics[width=0.75\textwidth]{Bilder/Pizza Toast.jpg}
\caption{Pizza Toast}
\end{figure}
\newpage
\section{Kräuertoast (Ida)}
\index{Kräuertoast (Ida)}
\RecipeMeta{1}{0}
\begin{ingredients}
  \item 
\end{ingredients}
\begin{directions}
  \item 
\end{directions}
\begin{figure}[h]
\centering
\includegraphics[width=0.75\textwidth]{Bilder/Kräuertoast (Ida).jpg}
\caption{Kräuertoast (Ida)}
\end{figure}
\newpage
\chapter{Suppen und Eintöpfe}
\section{Paprikacremesuppe}
\index{Paprikacremesuppe}
\RecipeMeta{4}{40}
\begin{ingredients}
  \item 700 g grüne Paprikaschoten
  \item 3 Knoblauchzehen
  \item 2 Schalotten
  \item 2 EL Butter
  \item 1 Rinderbouillon
  \item 100 g Crème double
  \item Salz
  \item Pfeffer
  \item Chilipulver
  \item 2 TL Essig
  \item Paprikawürfel (von 1 Paprikaschote)
  \item Petersilie
\end{ingredients}
\begin{directions}
  \item Paprikaschoten waschen, abtrocknen, vierteln und von Stielansätzen, Rippen und Kernen befreien. 1 Schote für die Würfel beiseitelegen, restliche Schoten in Streifen schneiden.
  \item Schalotten schälen und klein hacken, Knoblauch pressen.
  \item Butter in einem Suppentopf erhitzen, Schalotten glasig braten, Knoblauch dazugeben und Paprikastreifen 5 Minuten mitbraten.
  \item Mit Rinderbouillon ablöschen, zugedeckt 20 Minuten leicht kochen lassen.
  \item Die gegarten Paprikastreifen pürieren, zurück in den Topf geben, Crème double einrühren.
  \item Mit Salz, Pfeffer, Chilipulver und Essig abschmecken, erneut erhitzen.
  \item Mit Paprikawürfeln und Petersilie garnieren. Dazu passt frisches Brot mit Butter.
\end{directions}
\Notes{Die Suppe kann je nach Geschmack mit mehr Chilipulver scharf abgeschmeckt werden.}
\newpage
\section{Selleriesüppchen}
\index{Selleriesüppchen}
\RecipeMeta{1}{30}
\begin{ingredients}
  \item 1 Staude Bleichsellerie
  \item 1/2 Zitrone
  \item Salz
  \item Pfeffer
  \item Gemüsebrühe
  \item 1 Zwiebel
  \item etwas Butter
  \item etas Mehl
  \item 2 gekochte Kartoffeln
\end{ingredients}
\begin{directions}
  \item Aus Sellerie und Zitrone einen Sud dochen und abseihen
  \item Helle Einbrenne aus Zwiebel, BUtter und Mehl herstellen
  \item mit dem Sub aufgiesen und
  \item die Kartoffleln zugeben
\end{directions}
\begin{figure}[h]
\centering
\includegraphics[width=0.75\textwidth]{Bilder/Selleriesüppchen.jpg}
\caption{Selleriesüppchen}
\end{figure}
\newpage
\section{Pikante Tomaten-Käse-Suppe}
\index{Pikante Tomaten-Käse-Suppe}
\RecipeMeta{4}{50}
\begin{ingredients}
  \item 1 kg Fleischtomaten
  \item 4 mittelgroße Zwiebeln
  \item 4 EL Öl
  \item 1 TL Sambal Oelek (scharfe Würzpaste)
  \item 1 Knoblauchzehe
  \item schwarzer Pfeffer aus der Mühle
  \item 2 EL ungesalzene Sesamsamen
  \item 2 Scheiben Toastbrot
  \item 100 g würziger Edelpilzkäse
\end{ingredients}
\begin{directions}
  \item Tomaten über Kreuz einritzen, kurz in kochendes Wasser tauchen, häuten und ohne Stielansatz würfeln.
  \item Zwiebeln hacken.
  \item Öl in einem Topf erhitzen, Zwiebelwürfel glasig dünsten, Tomatenstücke zugeben.
  \item Mit Sambal Oelek würzen, zugedeckt bei milder Hitze ca. 20 Minuten kochen.
  \item Toastbrot würfeln, in einer Pfanne ohne Fett goldbraun rösten.
  \item Edelpilzkäse würfeln und unter die Suppe rühren.
  \item Mit Sesamsamen bestreuen und servieren.
\end{directions}
\Notes{Handschriftliche Notiz: 'Lecker! Aber süßer abschmecken, 8-10 min länger kochen. Ideal für Einladung!' Tipp: Wenn Sie sich das Häuten der Tomaten sparen möchten, fertige Suppe pürieren und durch ein Sieb streichen.}
\newpage
\section{Vichyssoise}
\index{Vichyssoise}
\RecipeMeta{4}{40}
\begin{ingredients}
  \item 250 g zarte Lauchstangen
  \item 200 g Kartoffeln (mehligkochend)
  \item 2 EL Olivenöl
  \item 1 Knoblauchzehe
  \item 2 TL frische Majoranblättchen
  \item 100 ml trockener Weißwein
  \item 1 l Hühnerbrühe
  \item 150 g Schmand (24\% Fett)
  \item Salz
  \item schwarzer Pfeffer aus der Mühle
  \item frisch geriebene Muskatnuss
  \item 1-2 EL Zitronensaft
  \item einige Tropfen Tabasco
\end{ingredients}
\begin{directions}
  \item Lauch putzen, waschen, ein kleines Stück in Frischhaltefolie wickeln und kühlen. Restlichen Lauch in feine Ringe schneiden.
  \item Kartoffeln waschen, schälen und in kleine Würfel schneiden.
  \item Öl erhitzen, Lauch und Kartoffeln darin andünsten. Knoblauch dazupressen, Majoran einrühren.
  \item Wein und Brühe angießen, aufkochen und zugedeckt bei milder Hitze ca. 30 Minuten köcheln lassen.
  \item Suppe pürieren, Schmand einrühren, mit Salz, Pfeffer, Muskat, Zitronensaft und Tabasco abschmecken.
  \item Über Nacht kühl stellen, am nächsten Tag nochmals durchrühren und abschmecken.
  \item Den gekühlten Lauch in Ringe schneiden und auf die Suppe streuen.
\end{directions}
\Notes{Handschriftliche Notiz: '15.11.2008 mit Sherry + etwas Sahne'. Tipp: Lauchcreme schmeckt auch heiß ausgezeichnet. Variante: Statt Schmand kann Crème fraîche oder Sahne verwendet werden.}
\newpage
\section{Chili con carne}
\index{Chili con carne}
\RecipeMeta{8}{150}
\begin{ingredients}
  \item 3 Zwiebeln
  \item 3 Knoblauchzehen
  \item 250 g Möhren
  \item 200 g Sellerie
  \item Paprika
  \item 750 g Tomaten
  \item 4 EL Öl
  \item 500 g Rinderhack
  \item 1/4 l Rotwein
  \item 1/4 l Brühe
  \item Thymian
  \item Chilipulver
  \item Salz, Pfeffer
  \item Kidneybohnen
  \item frische Kräuter
\end{ingredients}
\begin{directions}
  \item Gemüse hacken, Tomaten häuten.
  \item Hackfleisch anbraten, Zwiebeln und Knoblauch zugeben.
  \item Gemüse und Tomaten einrühren.
  \item Mit Wein und Brühe ablöschen, würzen.
  \item 1-1,5 Std. schmoren, Bohnen zugeben, weitere 15 Min. kochen.
\end{directions}
\Notes{Lecker, aber mild – nachwürzen!}
\newpage
\chapter{Auflauf}
\section{Pia (\& Lenos) Gemüse Lasagne }
\index{Pia (\& Lenos) Gemüse Lasagne }
\RecipeMeta{3}{60}
\begin{ingredients}
  \item Für rote Soße:
  \item Öl
  \item 2 Knoblauchzehen, gepresst
  \item 1 große Zwiebel, geschnitten
  \item 2 Paprika, gewürfelt
  \item 2 Möhren, gewüfelt
  \item passierte Tomaten
  \item Schlagsahne
  \item Chili-Pulver
  \item Curry-Pulver
  \item Kräutern der Provence
  \item Für grüne Soße:
  \item TK-Spinat
  \item Frischkäse, ca. 100g
  \item Hirtenkäse, gewürfelt
  \item Salz
  \item Pfeffer
  \item Muskatnuss
  \item Weitere Zutaten:
  \item Gauda, Mozzarella
  \item Lasagneplatten
\end{ingredients}
\begin{directions}
  \item Für die rote Soße:
  \item gewürfeltes GEmüse in Öl anbraten,
  \item mit Chili, Curry \& Kräutern der Provence würzen.
  \item Tomaten hinzugeben, aufkochen, Hitze reduzieren,
  \item Sahne hinzugeben, ein wenige köcheln lassen.
  \item Abschmeckenm bis die Soße ein bisschen zu scharf ist.
  \item Für die grüne Soße:
  \item Spinat auftauen, evtl. abtropfen,
  \item mit Frischkäse zum Kochen bringen.
  \item Hirtenkäse hinzugeben und weich werden lassen.
  \item Mit Salz, Pfeffer und Muskat würzen.
  \item Gauda reiben
  \item Mozzarella ein Scheiben schneiden.
  \item Lasagne wie folgt schichten. grün -> rot -> Platten -> wiederholen 
  \item mit Platten -> rot -> Käse abschließen
  \item Bei 180°C Umluft backen bis der Käse eine Kruste gebiltet hat.
\end{directions}
\begin{figure}[h]
\centering
\includegraphics[width=0.75\textwidth]{Bilder/Gemüse Lasagne(Pia und Leno).jpg}
\caption{Pia (\& Lenos) Gemüse Lasagne }
\end{figure}
\Notes{Backzeit ca. 20-25min}
\newpage
\section{Zucchini-Auflauf á la Mami}
\index{Zucchini-Auflauf á la Mami}
\RecipeMeta{4}{60}
\begin{ingredients}
  \item 250g Hackfleisch
  \item Zucchini
  \item Tomaten
  \item Zwiebeln
  \item 1 Ei
  \item geriebener Käse
  \item Knoblauch
  \item Chili
  \item Petersilie
  \item Salz
  \item Pfeffer
  \item Semmelbrösel
  \item Butter
\end{ingredients}
\begin{directions}
  \item Hackfleisch, Zucchini, Tomaten, Zwiebeln anbraten, dämpfen
  \item Ei, geriebener Käse, Knoblauch, Chili, Petersilie, Salz, Pfeffer unterrühren.
  \item mit Semmelbrösel und Bitter bestreuen
  \item 45min bei 220°C überbacken.
\end{directions}
\begin{figure}[h]
\centering
\includegraphics[width=0.75\textwidth]{Bilder/Zuchiniauflauf (Mama).jpg}
\caption{Zucchini-Auflauf á la Mami}
\end{figure}
\newpage
\section{Zucchini-Auflauf vegetarisch}
\index{Zucchini-Auflauf vegetarisch}
\RecipeMeta{1}{30}
\begin{ingredients}
  \item Zucchini
  \item Lauchzwiebeln
  \item Tomaten + - mark
  \item Champignons
  \item 299g Feta
  \item Salz
  \item Pfeffer
  \item Thymian
  \item Knoblauchöl
\end{ingredients}
\begin{directions}
  \item Zucchine aushölen, salzen und pfeffern
  \item Gemüse inkl. Zucchinirest anbraten und würzen
  \item zuletzt Käse in Würfeln zugeben
  \item ggf. gefüllte Hälfte mit dünnen Käsescheiben belegen
  \item 20-30 min bei 180°C Umlauf backen
\end{directions}
\begin{figure}[h]
\centering
\includegraphics[width=0.75\textwidth]{Bilder/Zuchiniauflauf Veggie.jpg}
\caption{Zucchini-Auflauf vegetarisch}
\end{figure}
\newpage
\chapter{Braten}
\section{Kaninchen in Pfeffersoße}
\index{Kaninchen in Pfeffersoße}
\RecipeMeta{4}{120}
\begin{ingredients}
  \item 1 Kaninchen
  \item Salz
  \item Pfeffer
  \item 3 EL Butterschmalz
  \item 1 Knoblauchzehe
  \item 2 Zwiebeln
  \item 700 ml heißes Wasser
  \item 100 ml süße Sahne
  \item 1 EL Mehl
  \item 2 EL grüner Pfeffer
\end{ingredients}
\begin{directions}
  \item Kaninchen waschen, trockentupfen und in portionsgerechte Stücke zerteilen. Mit Salz und Pfeffer einreiben.
  \item Butterschmalz erhitzen und Kaninchenstücke von allen Seiten anbraten.
  \item Zwiebeln und Knoblauch schälen, würfeln und zum Fleisch geben, kurz mit anbraten.
  \item Mit heißem Wasser ablöschen und Kaninchen ca. 1-1,5 Stunden schmoren lassen.
  \item Kaninchenstücke herausnehmen, warm halten. Fond durch ein Sieb gießen und aufkochen.
  \item Sahne und Mehl verquirlen, Fond damit binden. Grünen Pfeffer zugeben und Kaninchen in der Soße servieren.
\end{directions}
\Notes{Die Soße kann nach Belieben mit mehr Pfeffer abgeschmeckt werden.}
\newpage
\section{Rouladen}
\index{Rouladen}
\RecipeMeta{2}{120}
\begin{ingredients}
  \item 
\end{ingredients}
\begin{directions}
  \item 
\end{directions}
\begin{figure}[h]
\centering
\includegraphics[width=0.75\textwidth]{Bilder/Rouladen.jpg}
\caption{Rouladen}
\end{figure}
\newpage
\section{Schweinebraten}
\index{Schweinebraten}
\RecipeMeta{5}{180}
\begin{ingredients}
  \item 1kg Schweinebraten (nicht zu mager)
  \item Salz
  \item Pfeffer
  \item Fett zum anbraten
  \item 2 Lorbeerblätter
  \item einge Pfefferkörner
  \item einige Wacholderbeeren
  \item 1 zerdrückte Knoblauchzehe
  \item 1 Ziebel, grob geschnitten
  \item etwas Beifuß
\end{ingredients}
\begin{directions}
  \item Fleisch salzen und pfeffern und
  \item in Fett rundum anbraten
  \item Zwiebel zugeben und mitröten
  \item Mit Wasser ablöschen
  \item Gewürze zugeben
  \item Braten in die vorgezeizte Röhre schieben (ca. 220°C)
  \item ca. 1,5 - 2h garen
  \item Dazwischen öfter begießen
  \item evtl. mal wenden
  \item Braten rausnehmen, warmstellen
  \item Soße durchseihen, evtl. ergänzen
  \item abschmecken
  \item etwas sämig machen
\end{directions}
\begin{figure}[h]
\centering
\includegraphics[width=0.75\textwidth]{Bilder/Schweinebraten.jpg}
\caption{Schweinebraten}
\end{figure}
\newpage
\section{Weihnachtsgans}
\index{Weihnachtsgans}
\RecipeMeta{6}{300}
\begin{ingredients}
  \item 1 Gans (ca. 4kg)
  \item 1 Liter Wasser
  \item 2 Äpfel, in Spalten
  \item 1 Zwiebel, grob schneiden
  \item 2 Stengel Beifuß
\end{ingredients}
\begin{directions}
  \item Von der Gans die Flügel entfernen.
  \item Gans innen und außen kräfig salzen und pfeffern.
  \item Mit Äpfel, Zwiebel und Beifuß füllen und 
  \item alle Öffnungen zunähen.
  \item Mit der Brust nach unten in den Bräter auf den Einsatz legen.
  \item Gans mit kochendem Wasser übergießen und
  \item 1-1,5h zugedeckt kochen.
  \item Anschließend die Gans in die heiße Röhre geben. (nicht zudecken)
  \item Man brät die Gans in dem man sie ab und zu begießt.
  \item Ist die obere Seite braun, dreht man sie um und brät sie so fertig.
  \item (Insgesamt 1,5 - 2h)
  \item Letzte Viertelstunde die Gans nicht mehr begießen. Man spritzt etwas kaltes Wasser darauf, dann brät die Haut hart.
  \item Gans herausnehmen und warmstellen.
  \item Vor der Soße je nach Geschmack das Fett entfernen, evtl. den Rest mit Wasser ergänzen und etwas sämig machen.
\end{directions}
\begin{figure}[h]
\centering
\includegraphics[width=0.75\textwidth]{Bilder/Weinachtsgans 1.jpg}
\caption{Weihnachtsgans}
\end{figure}
\Notes{Falls die Gans gefroren ist 24h vorher bei Raumtemperatur auftauen.
Flügel mit dem Kragen und den Innereien zu einer Suppe kochen.}
\newpage
\section{Lammkeule auf provenzalische Art}
\index{Lammkeule auf provenzalische Art}
\RecipeMeta{5}{240}
\begin{ingredients}
  \item 1 Lammkeule (ca. 2 kg)
  \item Salz und Pfeffer
  \item 3 Knoblauchzehen
  \item 3 Zweige Rosmarin
  \item 200 ml Rotwein und 400 ml Brühe
  \item 50 g kleine festkochende Kartoffeln
  \item 2 EL Olivenöl
  \item 2 EL Zitronensaft
  \item 2 EL Honig
\end{ingredients}
\begin{directions}
  \item Backofen auf 180 °C vorheizen. Lammkeule waschen, trockentupfen und würzen.
  \item Knoblauchzehen abziehen, in Stifte schneiden und mit Rosmarin in die Keule stecken.
  \item Lammkeule in eine Fettfangschale legen und auf der zweiten Schiene ca. 1 Stunde braten.
  \item Nach und nach Rotwein und Brühe angießen.
  \item Kartoffeln waschen und schälen, Schalotten abziehen. Beides auf dem Backblech verteilen.
  \item Mit Salz, Pfeffer und Zitronensaft würzen, darüber Olivenöl geben.
  \item Weitere 45 Minuten braten. Zur Bindung Honig über die Lammkeule streichen.
  \item Lammkeule mit Gemüse und Kartoffeln anrichten. Dazu z. B. grüne Bohnen und Bohnenböhnchen.
\end{directions}
\Notes{Honig sorgt für eine feine Glasur. Gemüse kann nach Wunsch variiert werden.}
\newpage
\chapter{Geflügelgerichte}
\section{Putenspießchen mit Erdnusssauce}
\index{Putenspießchen mit Erdnusssauce}
\RecipeMeta{4}{30}
\begin{ingredients}
  \item 1 Paprikaschote (rot)
  \item 1 Paprikaschote (gelb)
  \item 1 Paprikaschote (grün)
  \item 1 Bund Frühlingszwiebeln
  \item 3 EL Öl
  \item schwarzer Pfeffer
  \item 12 Putenspießchen (fertig aus dem Tiefkühlregal)
  \item 2 EL Erdnusscreme
  \item 1 TL Zitronensaft
  \item Sesamsamen zum Bestreuen
\end{ingredients}
\begin{directions}
  \item Paprikaschoten putzen, waschen und in 1 cm breite Streifen schneiden.
  \item Frühlingszwiebeln putzen und waschen.
  \item Paprikastreifen und Frühlingszwiebeln in Öl anbraten.
  \item Putenspießchen dazugeben und unter Wenden 8 Minuten braten.
  \item Erdnusssauce aus Erdnusscreme, Zitronensaft, Salz und Wasser anrühren, kräftig kochen.
  \item Gemüse und Fleisch auf Spieße stecken, mit Erdnusssauce beträufeln.
  \item Mit Sesamsamen bestreuen und sofort servieren.
\end{directions}
\Notes{Raffiniert und schnell zubereitet.}
\newpage
\section{Hühnerleber-Pfanne}
\index{Hühnerleber-Pfanne}
\RecipeMeta{4}{35}
\begin{ingredients}
  \item 300 g frische Hühnerleber
  \item 2 cl Sherry (ursprünglich Portwein)
  \item 2 Schalotten
  \item 1 Knoblauchzehe
  \item 1 Bund glatte Petersilie
  \item 2 EL Öl
  \item 1 EL Butter
  \item Salz
  \item schwarzer Pfeffer aus der Mühle
  \item 1 TL eingelegte grüne Pfefferkörner
  \item 1 Schuss trockener Rotwein
  \item 125 g Sahne
\end{ingredients}
\begin{directions}
  \item Hühnerleber in etwa 1 cm breite Stücke schneiden, mit Sherry begießen und kurz ziehen lassen.
  \item Schalotten, Knoblauch und Petersilie fein hacken.
  \item Öl in einer Pfanne erhitzen, Schalotten, Knoblauch und die Hälfte der Petersilie anbraten.
  \item Butter in der Pfannenmitte zerlassen, Hühnerleber portionsweise darin unter Rühren anbraten.
  \item Alles mischen, mit Salz, frisch gemahlenem Pfeffer und grünem Pfeffer würzen.
  \item Den aufgefangenen Sherry, Rotwein und Sahne angießen, bei mittlerer Hitze ca. 5 Minuten köcheln lassen.
  \item Mit restlicher Petersilie bestreuen und heiß servieren.
\end{directions}
\Notes{Beilage: Baguette oder Toast. Handschriftliche Notiz: 'Stark! 12.01.02 als Hauptgericht für zwei.'}
\newpage
\section{Schnelles Puten-Chili}
\index{Schnelles Puten-Chili}
\RecipeMeta{4}{35}
\begin{ingredients}
  \item 600 g Putenbrust
  \item 2 EL Sojasauce
  \item 2 EL Zitronensaft
  \item 4 EL Öl
  \item Pfeffer
  \item Chilipulver
  \item Frühlingszwiebeln
  \item Kidneybohnen
  \item Tomatenmark
  \item Hühnerfond
  \item Salz
\end{ingredients}
\begin{directions}
  \item Putenfleisch würfeln, marinieren.
  \item Zwiebeln hacken, Bohnen abtropfen.
  \item Fleisch anbraten, Bohnen und Tomatenmark zugeben.
  \item Mit Fond angießen, 15 Min. garen.
  \item Mit Salz, Pfeffer und Chili abschmecken.
\end{directions}
\Notes{Schnell \& lecker, wieder machen!}
\newpage
\section{Hähnchenbrust-Curry}
\index{Hähnchenbrust-Curry}
\RecipeMeta{4}{50}
\begin{ingredients}
  \item Hähnchenbrustfilets
  \item Zwiebel
  \item Knoblauch
  \item Ingwer
  \item Öl
  \item Kurkuma
  \item Koriander
  \item Kreuzkümmel
  \item Cayennepfeffer
  \item Hühnerbrühe
  \item Joghurt
  \item Sahne
  \item Tomaten
  \item Petersilie
  \item Zitrone
\end{ingredients}
\begin{directions}
  \item Fleisch würzen, anbraten.
  \item Zwiebel, Knoblauch, Ingwer dünsten, Gewürze zugeben.
  \item Mit Brühe ablöschen, Joghurt und Sahne einrühren.
  \item Tomaten und Petersilie zugeben, 20 Min. garen.
  \item Mit Zitronenschnitzen servieren.
\end{directions}
\Notes{Super-mega-lecker!}
\newpage
\chapter{Fischgerichte}
\section{Rollfisch}
\index{Rollfisch}
\RecipeMeta{1}{60}
\begin{ingredients}
  \item 1 Goldbarschfilet (200g)
  \item 1 Tomate
  \item 1 kleine Zwiebel
  \item 1/2 Apfel
  \item Zitronensaft
  \item Salz
  \item Pfeffer
  \item Petersilie
  \item Tomatenmark
  \item Butter
  \item Öl
\end{ingredients}
\begin{directions}
  \item Fischfilets: 
  \item waschen, 
  \item abtufen,
  \item mit Zitronensatz beträufeln und 
  \item danach aufeinandergelegt 1/2h ziehen lassen
  \item Tomaten:
  \item brühen, häuten und in dicke Scheiben schneiden
  \item Ziebeln in dünne Scheiben schneiden
  \item Äpfel schälen, grob raspeln
  \item Feuerfeste Form einfetten
  \item Schichtung in Feuerfeste Form:
  \item Tomaten - Zwiebeln - Äpfel - Salz/Peffer
  \item Fischfilets: leicht saltzen/pfeffern und mit Tomatenmark bestreichen
  \item Butterflocken, gehackte Petersilie
  \item -> rollen
  \item keine Flüssigkeit zugeben;
  \item mit gefetteten Pergamentpapier abdecken
  \item 20min bei 200°C im vorgeheizten Ofen garen
\end{directions}
\begin{figure}[h]
\centering
\includegraphics[width=0.75\textwidth]{Bilder/Rollfisch.JPG}
\caption{Rollfisch}
\end{figure}
\newpage
\section{Gemüsetörtchen mit Lachs}
\index{Gemüsetörtchen mit Lachs}
\RecipeMeta{12}{20}
\begin{ingredients}
  \item 3 Zucchini
  \item 6 mittelgroße Möhren
  \item 500 g Champignons
  \item 60 g Emmentaler Käse
  \item 2 Eier
  \item 2 EL Olivenöl (kaltgepresst)
  \item 100 g Weizen- oder Dinkelvollkornmehl
  \item 1 TL Instant-Gemüsebrühe
  \item Salz, Pfeffer
  \item 100 g frischer Lachs
\end{ingredients}
\begin{directions}
  \item Gemüse und Käse mit der Trommel des Gemüseschneiders grob zerkleinern.
  \item Mehl, Gemüsebrühe und Gewürze mit dem Flachrührer unterrühren.
  \item Eier und Öl hinzufügen und alles zu einem Teig vermischen (ca. 5 Minuten).
  \item Den Teig in Muffinförmchen füllen.
  \item Lachs in kleine Stücke schneiden und auf die Muffins legen.
  \item Mit dem restlichen Käse bestreuen.
  \item Im vorgeheizten Backofen bei 180 °C ca. 20 Minuten backen.
  \item Ergibt ca. 12 Stück.
\end{directions}
\Notes{Variante: Tomaten mit dem Pürieraufsatz der Gemüsepresse pürieren und hinzufügen.}
\newpage
\section{Rotbarsch in Kräutersauce}
\index{Rotbarsch in Kräutersauce}
\RecipeMeta{4}{45}
\begin{ingredients}
  \item 600 g Rotbarschfilets
  \item 1 Zitrone
  \item Salz, Pfeffer
  \item 1 Bund Basilikum
  \item 1 Bund Petersilie
  \item 2 Zweige Thymian
  \item 1 Zwiebel
  \item 1 EL Butterschmalz
  \item 100 ml Weißwein
  \item 100 ml Brühe
  \item 100 g Frischkäse
  \item 100 g Kirschtomaten
\end{ingredients}
\begin{directions}
  \item Fisch würfeln, mit Zitronensaft beträufeln.
  \item Zwiebel hacken, in Butterschmalz dünsten.
  \item Kräuter zugeben, mit Wein und Brühe ablöschen.
  \item Frischkäse einrühren, würzen.
  \item Fischstücke in Sauce ziehen lassen, Tomaten zugeben.
\end{directions}
\Notes{Wieder machen! Tipp: Für Kinder Wein durch Brühe ersetzen.}
\newpage
\chapter{Süßspeißen}
\section{Pfannkuchen}
\index{Pfannkuchen}
\RecipeMeta{4}{20}
\begin{ingredients}
  \item 200g Mehl
  \item 2 EL Zucker
  \item 2 TL Backpulver
  \item 1 Prise Salz
  \item 300 ml Milch
  \item 1-2 Eier
  \item 2 EL geschmolzene Butter + Butter zum Braten
\end{ingredients}
\begin{directions}
  \item Zutaten vermischen
  \item 2-3 EL pro Pfannkuchen in die Pfanne geben
  \item Warm servieren
\end{directions}
\Notes{Ganz einfach}
\newpage
\section{Quarkstollen}
\index{Quarkstollen}
\RecipeMeta{25}{90}
\begin{ingredients}
  \item 500 g Mehl
  \item 1 Päckchen Backpulver
  \item 1 Prise Salz
  \item 1 TL Kardamom
  \item abgeriebene Schale einer halben Zitrone (ungespritzt)
  \item 50 g Orangeat (gewürfelt)
  \item 50 g Sukkade (gewürfelt)
  \item 75 g gehackte Mandeln
  \item 125 g Rosinen
  \item 125 g Korinthen
  \item 200 g Zucker
  \item 250 g Magerquark (gut ausgedrückt)
  \item 175 g Sanella oder Butter in Flöckchen
  \item 2 Eier
  \item 2 EL Rum
  \item 50 g zerlassene Sanella oder Butter
  \item Puderzucker zum Bestäuben
\end{ingredients}
\begin{directions}
  \item Mehl, Backpulver, Salz und Kardamom auf ein Backbrett geben und vermischen.
  \item Zitronenschale, Orangeat, Sukkade, Mandeln, Rosinen, Korinthen und Zucker daraufgeben.
  \item Magerquark und Sanella in Flöckchen darüber verteilen, eine Mulde eindrücken.
  \item Eier, Rum und zerlassene Sanella hineingeben.
  \item Alles mit einem Messer durcharbeiten, dann rasch zusammenkneten.
  \item Einen Stollen formen und auf ein gefettetes Blech legen.
  \item Im vorgeheizten Ofen bei 175-200 °C (Gas 2-3) ca. 60 Minuten backen.
  \item Noch heiß mit Butter oder Sanella bepinseln und mit Puderzucker bestäuben.
\end{directions}
\Notes{Saftiger Stollen mit Quark statt Hefe.}
\newpage
\section{Indischer Milchreis}
\index{Indischer Milchreis}
\RecipeMeta{1}{30}
\begin{ingredients}
  \item 250g Rundkornreis
  \item 1 Liter Milch
  \item 3 Nelken
  \item 1 Prise Kardamom
  \item 1 Prise Koriander
  \item 1 Prise Salz
  \item 100g Zucker
\end{ingredients}
\begin{directions}
  \item Reis mit der Milch und den restlichen Zutaten aufkochen und ziehen lassen.
\end{directions}
\begin{figure}[h]
\centering
\includegraphics[width=0.75\textwidth]{Bilder/Indischer Milchreis.jpg}
\caption{Indischer Milchreis}
\end{figure}
\Notes{Am besten während das Aufkochens die Milch beobachten und regelmässig umrühren. Danach den Topf von der Hitze nehmen und zugedeckt ca. 20 min ziehen lassen.}
\newpage
\section{Apfelstrudel – schnelle Art}
\index{Apfelstrudel – schnelle Art}
\RecipeMeta{1}{30}
\begin{ingredients}
  \item 1 Grundrezept Blätterteig oder 2 P. Tiefkühl-Blätterteig
  \item 30 g gemahlene Haselnüsse
  \item 1 kg Äpfel
  \item abgeriebene Schale 1 Zitrone (ungespritzt)
  \item 3 EL Zitronensaft
  \item 100 g Zucker
  \item 1 TL Zimt
  \item 30 g gemahlene Haselnüsse
  \item evtl. 30 g Sultaninen
  \item zerlassene Sanella
  \item Puderzucker
\end{ingredients}
\begin{directions}
  \item Blätterteig ausrollen (40 x 50 cm) und mit Haselnüssen bestreuen.
  \item Äpfel schälen und in feine Scheibchen schneiden.
  \item Mit Zitronenschale, Zitronensaft, Zucker, Zimt, Haselnüssen und evtl. Sultaninen vermengen.
  \item Füllung auf den Teig geben, Ränder etwas einschlagen, anfeuchten und locker aufrollen.
  \item Mit der Naht nach unten auf ein kalt abgespültes Blech legen und mit zerlassener Sanella bepinseln.
  \item Oberfläche mit der Schere einschneiden.
  \item Im vorgeheizten Ofen bei 225–250 °C ca. 30 Minuten backen.
  \item Noch heiß mit Puderzucker bestäuben. Warm oder kalt servieren.
\end{directions}
\Notes{Ergibt ca. 10 Stück. Backzeit: ca. 30 Minuten.}
\newpage
\chapter{Beilagen}
\section{Fladenbrot}
\index{Fladenbrot}
\RecipeMeta{4}{60}
\begin{ingredients}
  \item 300ml Milch
  \item 550g Mehl
  \item 1 Päckchen Trockenhefe
  \item 1/2 TL Zucker
  \item 1 Prise Salz
  \item 2 Eier
  \item Für das Backblech:
  \item 1 TL Olivenöl
  \item Zum Bestreuen:
  \item 2 EL Sesamsamen
  \item 1 TL Schwarzkümmel
\end{ingredients}
\begin{directions}
  \item 500g Mehl in eine Schüssel sieben
  \item Hefe, Zucker und Saltz darunter mischen
  \item 250ml warme Milch + 1 Ei einkneten
  \item 30min an einem warmen Ort gehen lassen
  \item danach mit etwas Mehl den Teig in 4 Stücke teilen und
  \item als Fladen auf das eingelöte Backblech ausrollen
  \item 1 Ei, 2 EL Milch, 1/2 TL Zucker und etwas Olivenöl verquirlen
  \item Teig rautenförmig eindrücken
  \item Fladen bepinseln und mit Sesam und Kümmel betreuen
  \item ca. 10 min bei 275°C Umluft backen.
\end{directions}
\begin{figure}[h]
\centering
\includegraphics[width=0.75\textwidth]{Bilder/Fladenbrot.jpg}
\caption{Fladenbrot}
\end{figure}
\newpage
\section{Grüne Klöße}
\index{Grüne Klöße}
\RecipeMeta{5}{60}
\begin{ingredients}
  \item 20 Kartoffeln, mehlig kochende
  \item Salz
  \item Wasser
  \item Knödelhilfe
  \item Kartoffelmehl
  \item gerötete Weißbrotwürfel ("Greebala")
\end{ingredients}
\begin{directions}
  \item 2/3 der Kartoffen in etwas Wasser und Knödekhilfein eine Schüssel reiben.
  \item Gut durchmischen
  \item Den Rest der Kartoffen schälen und kochen
  \item Die geriebenen Kartoffeln gut auspressen,
  \item in eine Schüssel geben und auflockern und salzen.
  \item Mit kochendem Wasser brühen (nicht zuviel sonst wird der Teig zu weich)
  \item Gekochte Kartoffeln durch die Presse darauf geben
  \item ca. 2 EL Kartoffelmehl überstreuen und 
  \item das Ganze erst mit Holzlöffel dann mit den Händen kräftig durchkneten.
  \item Mit nassen Händen Klöße formen.
  \item In die Mitte der Klöße "Greebala" geben.
  \item Die Klöße in reichlich kochendes Salzwasser gleiten lassen 
  \item Temperatur reduzieren und
  \item in ca. 25min fertig ziehen lassen.
\end{directions}
\begin{figure}[h]
\centering
\includegraphics[width=0.75\textwidth]{Bilder/Klöße Grün1.jpg}
\caption{Grüne Klöße}
\end{figure}
\Notes{Das Wasser darf nur am Anfang sprudeln. Zum Ziehen lassen den Topf nicht zudecken!}
\newpage
\section{gekochte Klöße ala Papa}
\index{gekochte Klöße ala Papa}
\RecipeMeta{5}{60}
\begin{ingredients}
  \item 1kg mehlig Kartoffeln
  \item 1/2 TL Salz
  \item 3-4 EL Kartoffelmehl
  \item 1 EL Mehl
  \item 1-2 Eier
\end{ingredients}
\begin{directions}
  \item Kartoffeln 15-20min kochen, 
  \item schälen, 
  \item pressen und 
  \item mit Ei, Salz, Mehl und Kartoffelmehl vermengen.
  \item Knödel formen
  \item 10-15min in leicht köchelndem Salzwasser ziehen lassen.
\end{directions}
\begin{figure}[h]
\centering
\includegraphics[width=0.75\textwidth]{Bilder/Klöße.jpg}
\caption{gekochte Klöße ala Papa}
\end{figure}
\Notes{Eier machen den Teig weicher. Daher sparsam mit den Eier umgehen. Wasser sollte nachdem die Knödel drin sind, nicht mehr sprudeln. Temperatur reduzieren! Natürlich kommen auch hier geröstetes Brot in die Klöße. Es muss aber kein Weißbrot sein!}
\newpage
\section{Rahmchampignions}
\index{Rahmchampignions}
\RecipeMeta{2}{30}
\begin{ingredients}
  \item 400g frische Champignons
  \item 1 große Zwiebel
  \item Salz
  \item Pfeffer
  \item gehackte Petersilie
  \item 150g Creme fraiche
  \item Weißwein
  \item Butter
\end{ingredients}
\begin{directions}
  \item Zwiebeln mit Butter andünsten, 
  \item Pilze und Wein zugeben.
  \item Mit geschlossenem Deckel dünsten;
  \item zuletzt Rahm und Kräuter zugeben.
  \item Mit Salz und Pfeffer abschmecken.
\end{directions}
\begin{figure}[h]
\centering
\includegraphics[width=0.75\textwidth]{Bilder/Rahmchampingions.jpg}
\caption{Rahmchampignions}
\end{figure}
\Notes{Als Beilage zu Fleisch oder Hauptgericht mit Reis.}
\newpage
\section{Spargel}
\index{Spargel}
\RecipeMeta{4}{15}
\begin{ingredients}
  \item 1 kg Spargel
  \item Wasser
  \item 1 TL Salz
  \item 2 TL Zucker
  \item Für den Sud zum Einlegen:
  \item 3-4 EL Essig
  \item 1 TL Zucker
  \item Pfeffer, Petersilie, Schnittlach, Dill
  \item 1 EL Öl
  \item etwas Salz
\end{ingredients}
\begin{directions}
  \item Spargel schälen
  \item Spargel 15 min in leicht bedeckten Salzwasser kochen
  \item Warm in den Sud legen und
  \item Warm oder kalt servieren
\end{directions}
\begin{figure}[h]
\centering
\includegraphics[width=0.75\textwidth]{Bilder/Spargel.jpg}
\caption{Spargel}
\end{figure}
\Notes{Hält mehrere Tage im Sud, wenn man keine Mitesser hat :-)}
\newpage
\section{Tiroler Speckknödel}
\index{Tiroler Speckknödel}
\RecipeMeta{1}{45}
\begin{ingredients}
  \item 6 altbackene Brötchen
  \item 3 Eier
  \item 200 ml Milch
  \item 2 Zwiebeln
  \item 100 g Speck
  \item 1 EL gehackte Petersilie
  \item 2-3 EL Mehl
  \item Salz
\end{ingredients}
\begin{directions}
  \item Brötchen in Würfel schneiden und in eine Schüssel geben.
  \item Milch mit den Eiern verquirlen und über die Brotwürfel gießen.
  \item Speck in kleine Würfel schneiden und in einer Pfanne auslassen.
  \item Zwiebeln fein würfeln, zu dem Speck in die Pfanne geben und glasig werden lassen.
  \item Vom Herd nehmen und abkühlen lassen.
  \item Speck-Zwiebel-Mischung und Petersilie zu den Brotwürfeln geben und alles zu einem Teig verkneten.
  \item Zugedeckt 30 Minuten ruhen lassen.
  \item Falls der Teig klebrig ist, etwas Mehl unterkneten.
  \item Aus dem Teig 8 große Knödel formen.
  \item In einem großen Topf Salzwasser zum Kochen bringen, Knödel hineingeben, einmal aufkochen und bei kleiner Hitze 15-20 Minuten ziehen lassen.
  \item Mit gebräunter Butter übergießen und zu Sauerkraut oder Weißkraut servieren.
\end{directions}
\Notes{Zubereitungszeit inkl. Ruhezeit: ca. 45 Minuten. Nährwerte pro Portion: 370 kcal, 19 g Eiweiß, 21 g Fett, 25 g Kohlenhydrate.}
\newpage
\section{Marinierte Zwiebeln}
\index{Marinierte Zwiebeln}
\RecipeMeta{6}{30}
\begin{ingredients}
  \item 500 Perlzwiebeln
  \item 5 EL Olivenöl
  \item 1 Knoblauchzehe
  \item 1 unbehandelte Zitrone
  \item 50 g Sultaninen
  \item 1/2 Glas Weißwein
  \item 1 Zweig Thymian
  \item 1 EL Weinessig
  \item 1 Lorbeerblatt
  \item Salz
  \item 1/2 TL Zucker
  \item Petersilie oder Koriander
\end{ingredients}
\begin{directions}
  \item Perlzwiebeln schälen und in Olivenöl andünsten.
  \item Knoblauch, Zitronenscheiben, Sultaninen, Weißwein, Thymian, Essig und Lorbeer hinzufügen.
  \item 15-20 Minuten zugedeckt kochen, bis Zwiebeln weich sind.
  \item Zitronenscheiben und Thymian entfernen, abkühlen lassen.
  \item Mit Salz, Zucker und gehackten Kräutern bestreuen.
\end{directions}
\Notes{Ideal als Vorspeise oder Beilage.}
\newpage
\chapter{Herzhafte Kuchen}
\section{Zwiebelkuchen}
\index{Zwiebelkuchen}
\RecipeMeta{6}{60}
\begin{ingredients}
  \item Mehl
  \item Hefe
  \item Zucker
  \item Öl
  \item Salz
  \item Zwiebeln
  \item Butterschmalz
  \item Pfeffer
  \item Majoran
  \item Crème fraîche
  \item Eier
  \item Speck
  \item Sonnenblumenkerne
\end{ingredients}
\begin{directions}
  \item Teig herstellen und gehen lassen.
  \item Zwiebeln dünsten, würzen.
  \item Backofen vorheizen, Teig ausrollen.
  \item Belag auf Teig verteilen, backen.
\end{directions}
\Notes{Schmeckt frisch oder lauwarm, kann eingefroren werden.}
\newpage
\chapter{Kuchen, Torten und Gebäck}
\section{Nürnberger Busserl}
\index{Nürnberger Busserl}
\RecipeMeta{1}{30}
\begin{ingredients}
  \item 500 g Farinzucker
  \item 3 EL Honig
  \item 4 Eier
  \item 20 g Zimt
  \item 525 g Mehl
  \item 1 Päckchen Backpulver
  \item 100 g Zitronat oder Orangat
  \item 100g gemahlene Mandeln oder Nüsse
  \item Honigwasser zum Bestreichen
\end{ingredients}
\begin{directions}
  \item Farinzucker, Honig und Eier schaumig rühren.
  \item Zimt, Mehl, Backpulver, Zitronat und Mandeln oder Nüsse hinzufügen und vermengen.
  \item Mit Knethaken einen Teig herstellen.
  \item Mit leicht bemehlten Händen kleine Kugeln formen.
  \item Auf gefettetes Blech oder Backpapier legen und mit Honigwasser bestreichen.
  \item Bei Mittelhitze (150°C) 15-20 Minuten backen.
  \item Noch heiß nochmals mit Honigwasser bestreichen.
\end{directions}
\Notes{Kleine Honigplätzchen, typisch weihnachtlich.}
\newpage
\section{Hildatörtchen}
\index{Hildatörtchen}
\RecipeMeta{65}{45}
\begin{ingredients}
  \item 500 g Mehl
  \item 250 g Zucker
  \item 1 Päckchen Vanillin-Zucker
  \item abgeriebene Schale von 1 Zitrone
  \item 3 Eigelb
  \item 250 g Butter
  \item Marmelade zum Füllen
  \item Puderzucker zum Bestäuben
\end{ingredients}
\begin{directions}
  \item Mehl sieben, mit Zucker und Vanillin-Zucker mischen.
  \item Zitronenschale abreiben und hinzufügen.
  \item In der Mitte eine Vertiefung machen, Eigelb und Butter hineingeben.
  \item Die Masse verkneten und kalt stellen.
  \item Den Teig messerrückendick ausrollen.
  \item Kleine Plätzchen ausstechen und bei Mittelhitze ca. 15 Minuten backen.
  \item Je zwei Plätzchen mit Marmelade zusammensetzen.
  \item Mit Puderzucker übersieben.
\end{directions}
\Notes{Klassische Doppelkekse mit Marmeladefüllung.}
\newpage
\section{Nürnberger Lebkuchen}
\index{Nürnberger Lebkuchen}
\RecipeMeta{20}{40}
\begin{ingredients}
  \item 300 g Puderzucker
  \item 1 Päckchen Vanillin-Zucker
  \item 4 Eier
  \item 65 g gehacktes Zitronat
  \item 65 g gehacktes Orangeat
  \item 250 g gesiebtes Mehl
  \item 1 Päckchen Lebkuchengewürz
  \item 1 Messerspitze Backpulver
  \item 200 g Mandeln
  \item Oblaten
\end{ingredients}
\begin{directions}
  \item Puderzucker, Vanillin-Zucker und Eier schaumig rühren.
  \item Zitronat, Orangeat, Mehl, Gewürz und Backpulver mischen und unter die Masse heben.
  \item Mandeln zugeben und alles gut verrühren.
  \item Masse auf Oblaten streichen (1 gehäufter Teelöffel pro Oblate) und bei schwacher Hitze 15-20 Minuten backen.
\end{directions}
\Notes{Schwache Hitze verwenden, damit die Lebkuchen nicht zu dunkel werden.}
\newpage
\section{Bischofshütchen}
\index{Bischofshütchen}
\RecipeMeta{30}{45}
\begin{ingredients}
  \item 250 g Mehl
  \item 65 g Zucker
  \item 1 Ei
  \item 125 g Butter oder Margarine
  \item 1 Eiweiß
  \item 50 g Zucker
  \item 70 g geriebene Nüsse
\end{ingredients}
\begin{directions}
  \item Mürbteig herstellen: Mehl sieben, Zucker und Ei in die Mitte geben, Butter/Margarine zerschneiden und zu einem glatten Teig kneten. Kalt stellen.
  \item Teig dünn ausrollen, kleine runde Plätzchen ausstechen.
  \item Eiweiß steif schlagen, Zucker dazugeben, geriebene Nüsse unterheben.
  \item Plätzchen mit 1 TL Makronenmasse belegen, bei 180-200°C 20-25 Minuten backen.
\end{directions}
\Notes{Makronenmasse vorsichtig unterheben, damit die Masse luftig bleibt.}
\newpage
\section{Ischler Plätzchen}
\index{Ischler Plätzchen}
\RecipeMeta{30}{50}
\begin{ingredients}
  \item 140 g Mehl
  \item 70 g Zucker
  \item 70 g geriebene Mandeln
  \item 140 g Butter oder Margarine
  \item etwas Zitronensaft
  \item Marmelade
  \item Schokoladenglasur
  \item gehackte Pistazien oder Mandeln
\end{ingredients}
\begin{directions}
  \item Mürbteig herstellen: Mehl, Zucker, Mandeln mischen, Butter und Zitronensaft zugeben, zu einem glatten Teig kneten. Kalt stellen.
  \item Teig 0,5 cm dick ausrollen, Plätzchen ausstechen und bei mittlerer Hitze ca. 15 Minuten backen.
  \item Plätzchen mit Marmelade aufeinander setzen, oben und seitlich mit Schokoladenglasur bestreichen und nach Belieben mit Pistazien oder Mandeln bestreuen.
\end{directions}
\Notes{Kalt lagern, schmecken nach einigen Tagen noch besser.}
\newpage
\section{Spitzbuben}
\index{Spitzbuben}
\RecipeMeta{60}{50}
\begin{ingredients}
  \item 420 g Mehl
  \item 210 g Zucker
  \item 125 g geriebene Haselnüsse
  \item 250 g Butter oder Margarine
  \item 1 Ei
  \item Marmelade
\end{ingredients}
\begin{directions}
  \item Mürbteig herstellen: Mehl, Zucker, Nüsse mischen, Butter und Ei dazugeben, zu einem glatten Teig kneten. 1 Stunde kaltstellen.
  \item Teig 0,5 cm dick ausrollen, Plätzchen ausstechen und bei mittlerer Hitze 15-20 Minuten backen.
  \item Je 2 Plätzchen mit Marmelade zusammensetzen.
\end{directions}
\Notes{Plätzchen lassen sich gut einfrieren.}
\newpage
\section{Zimtsterne}
\index{Zimtsterne}
\RecipeMeta{40}{40}
\begin{ingredients}
  \item 2 Eiweiß
  \item 250 g Zucker
  \item 2-4 TL Zimt
  \item 300 g gemahlene Mandeln
  \item 1 Eiweiß
  \item 50 g Puderzucker
\end{ingredients}
\begin{directions}
  \item Eiweiß sehr steif schlagen, Zucker und Zimt einrühren.
  \item Mandeln langsam unterheben, Teig kaltstellen.
  \item Teig 1 cm dick ausrollen, Sternchen ausstechen, auf Backpapier setzen.
  \item Baiser aus 1 Eiweiß und 50 g Puderzucker herstellen, Plätzchen damit bestreichen.
  \item Bei schwacher Hitze ca. 20 Minuten backen, mehr trocknen als backen.
\end{directions}
\Notes{Schwache Hitze verwenden, damit die Baiserhaube nicht bräunt.}
\newpage
\section{Nougatecken}
\index{Nougatecken}
\RecipeMeta{30}{45}
\begin{ingredients}
  \item 250 g Weizenmehl
  \item 1 gestrichener TL Backpulver
  \item 75 g Zucker
  \item 1 Päckchen Vanillin-Zucker
  \item 1 Ei
  \item 200 g Margarine
  \item 150 g gemahlene Haselnüsse
  \item 200 g Nougatmasse
  \item 50-75 g Kuvertüre
\end{ingredients}
\begin{directions}
  \item Mehl und Backpulver mischen, Zucker, Vanillin-Zucker und Ei einarbeiten.
  \item Margarine und Haselnüsse daraufgeben, alle Zutaten schnell zu einem glatten Teig verkneten, kaltstellen.
  \item Teig dünn ausrollen, Ecken ausstechen, bei mittlerer Hitze ca. 10 Minuten backen.
  \item Nougatmasse im Wasserbad erwärmen, Hälfte der Plätzchen damit bestreichen, andere Hälfte darauflegen.
  \item Kuvertüre im Wasserbad schmelzen, Enden der Plätzchen eintauchen.
\end{directions}
\Notes{Nougatmasse vorher glatt rühren, nicht zu heiß verwenden.}
\newpage
\section{Kirsch-Nuss-Kuchen}
\index{Kirsch-Nuss-Kuchen}
\RecipeMeta{8}{75}
\begin{ingredients}
  \item 500 g Kirschen, entsteint
  \item 150 g Butter
  \item 150 g Zucker
  \item 4 Eier
  \item 100 g Schokolade, gerieben
  \item 200 g gemahlene Haselnüsse
\end{ingredients}
\begin{directions}
  \item Butter und Zucker schaumig rühren.
  \item Eier trennen. Eigelbe nacheinander unter die Buttermasse rühren.
  \item Schokolade und gemahlene Haselnüsse hinzufügen.
  \item Eiweiße steif schlagen und unterheben.
  \item Teig in eine Springform füllen, Kirschen darauf verteilen.
  \item Im vorgeheizten Ofen bei 180°C auf der unteren Schiene 1 Stunde backen.
  \item Kuchen 10 Minuten im ausgeschalteten Ofen auskühlen lassen, anschließend mit Puderzucker bestreuen.
\end{directions}
\Notes{Eier trennen und Eiweiß steif schlagen für lockere Konsistenz.}
\newpage
\section{Ostpreußischer Pfefferkuchen}
\index{Ostpreußischer Pfefferkuchen}
\RecipeMeta{12}{60}
\begin{ingredients}
  \item 500 g Honig
  \item 500 g Zucker
  \item 200 g Butter oder Margarine
  \item 1500 g Mehl
  \item 200 g gemahlene Haselnüsse
  \item 100 g gehacktes Zitronat
  \item 2 EL Kakao
  \item 2 TL Neunerlei-Gewürz 2 x 15g = 30g
  \item 1 Päckchen Backpulver 1,5 x 15g = 22,5g
  \item 6 Eier
\end{ingredients}
\begin{directions}
  \item Honig, Zucker, Butter/Margarine zusammen erwärmen und verrühren, dann abkühlen lassen.
  \item Mehl, Nüsse, Zitronat, Kakao, Gewürz, Backpulver mischen.
  \item Abgekühlte Honigmasse dazugeben und alles gut verkneten.
  \item Teig kaltstellen, ca. 1 cm dick ausrollen, Plätzchen ausstechen.
  \item Bei mittlerer Hitze 15-20 Minuten backen.
\end{directions}
\Notes{Gewürzkuchen entfaltet sein volles Aroma erst nach einem Tag.}
\newpage
\section{Gewürzkuchen}
\index{Gewürzkuchen}
\RecipeMeta{8}{70}
\begin{ingredients}
  \item 200 g weiche Butter
  \item 250 g Zucker
  \item 2 TL Vanillezucker
  \item 4 Eier
  \item 200 g gemahlene Haselnüsse
  \item 1 TL Zimtpulver
  \item je 1 Messerspitze Gewürznelken- und Ingwerpulver
  \item 250 g Mehl
  \item 3 TL Backpulver
  \item 2 EL Rum
\end{ingredients}
\begin{directions}
  \item Butter, Zucker und Vanillezucker schaumig rühren.
  \item Eier nach und nach unterrühren.
  \item Nüsse, Gewürze und Mehl mit Backpulver mischen und unter den Teig heben.
  \item Rum unterrühren, Teig in Kuchenform füllen.
  \item Bei 200°C auf mittlerer Schiene 1 Stunde backen.
\end{directions}
\Notes{Teig entfaltet erst nach einem Tag sein volles Aroma.}
\newpage
\section{Klassischer Marmorkuchen}
\index{Klassischer Marmorkuchen}
\RecipeMeta{8}{70}
\begin{ingredients}
  \item 250 g Butter oder Margarine
  \item 250 g Zucker
  \item 1 Päckchen Vanillezucker
  \item 4 Eier
  \item 400 g Mehl
  \item 100 g Speisestärke
  \item 1 Päckchen Backpulver
  \item 1 Prise Salz
  \item 100 ml Milch
  \item 30 g Kakaopulver
\end{ingredients}
\begin{directions}
  \item Butter, Zucker und Vanillezucker schaumig rühren, Eier unterrühren.
  \item Mehl, Speisestärke, Backpulver und Salz mischen und abwechselnd mit Milch unter den Teig rühren.
  \item Hälfte des Teigs in vorbereitete Form füllen, Kakaopulver unter den restlichen Teig mischen und darüber geben.
  \item Mit einer Gabel leicht marmorieren, bei 180°C 1 Stunde backen.
  \item Mit Puderzucker bestreuen.
\end{directions}
\Notes{Garprobe mit Holzspieß durchführen.}
\newpage
\section{Tiroler Schokoladenkuchen}
\index{Tiroler Schokoladenkuchen}
\RecipeMeta{8}{80}
\begin{ingredients}
  \item 200 g weiche Butter oder Margarine
  \item 200 g Zucker
  \item 6 Eigelbe
  \item 125 g Mehl
  \item 1 TL Backpulver
  \item 200 g gemahlene Mandeln
  \item 150 g geraspelte Zartbitterschokolade
  \item 6 Eiweiße
  \item ein paar Tropfen Rum-Aroma
  \item 200 g Schokoladen-Fettglasur
\end{ingredients}
\begin{directions}
  \item Butter und Zucker schaumig rühren, Eigelbe unterrühren.
  \item Mehl mit Backpulver sieben, auf die Schaum-Masse geben.
  \item Mandeln und Schokolade unterheben.
  \item Eiweiße steif schlagen und mit Rum-Aroma unter den Teig ziehen.
  \item Teig in gebutterte und mit Semmelbrösel ausgestreute Form geben, glattstreichen, bei 180°C 1 Stunde backen.
  \item Schokoladenglasur im Wasserbad auflösen und Kuchen damit bestreichen.
\end{directions}
\Notes{Teig vorsichtig unterheben, um die Luftigkeit zu erhalten.}
\newpage
\section{Gefüllte Mandelstangen}
\index{Gefüllte Mandelstangen}
\RecipeMeta{30}{45}
\begin{ingredients}
  \item 200 g Weizenmehl
  \item 100 g Zucker
  \item 1 Päckchen Vanillin-Zucker
  \item 1 Prise Salz
  \item 1 Eigelb
  \item Eiweiß
  \item 100 g Margarine
  \item 75 g abgezogene, gehackte Mandeln
  \item Aprikosen-Konfitüre
  \item 100 g Kuvertüre
\end{ingredients}
\begin{directions}
  \item Knetteig zubereiten: Mehl sieben, Zucker, Vanillin-Zucker, Salz, Eigelb, Eiweiß und Margarine zu einem glatten Teig verkneten. Kalt stellen, falls klebrig.
  \item Teig dünn ausrollen, in Streifen (ca. 1 x 6 cm) schneiden, mit Eiweiß bestreichen und Mandeln bestreuen.
  \item Auf Backblech legen, bei mittlerer Hitze 10-15 Minuten backen.
  \item Hälfte der Stangen mit Aprikosen-Konfitüre bestreichen, die übrigen darauflegen.
  \item Kuvertüre im Wasserbad erwärmen, Enden der Stangen eintauchen.
\end{directions}
\Notes{Teig kaltstellen erleichtert das Ausrollen. Mandeln gleichmäßig verteilen.}
\newpage
\section{Gefüllte Schokoladenhörnchen}
\index{Gefüllte Schokoladenhörnchen}
\RecipeMeta{30}{50}
\begin{ingredients}
  \item 250 g Butter oder Margarine
  \item 150 g gesiebter Puderzucker
  \item 1 Päckchen Vanillin-Zucker
  \item 1 Fläschchen Rum-Aroma
  \item 1 Ei
  \item 175 g Weizenmehl
  \item 75 g Speisestärke
  \item 30 g Kakao
  \item 125 g Nougatmasse
  \item 125 g Kuvertüre
\end{ingredients}
\begin{directions}
  \item Butter, Puderzucker, Vanillin-Zucker, Rum-Aroma und Ei schaumig rühren.
  \item Mehl, Speisestärke, Kakao mischen, sieben und esslöffelweise unterrühren.
  \item Teig in Spritzbeutel mit glatter Lochtülle füllen, Hörnchen in Form kleiner Halbmonde oder Stangen auf Backblech spritzen.
  \item Bei 175-200°C ca. 10-15 Minuten backen.
  \item Nougatmasse im Wasserbad geschmeidig rühren, Hälfte der Plätzchen damit bestreichen, andere Hälfte fest dagegen drücken.
  \item Kuvertüre im Wasserbad schmelzen, Enden der Plätzchen eintauchen.
\end{directions}
\Notes{Hörnchen auf Pergamentpapier spritzen, damit sie nicht kleben.}
\newpage
\section{Nußstangen}
\index{Nußstangen}
\RecipeMeta{30}{40}
\begin{ingredients}
  \item 130 g Butter
  \item 2 Eigelb
  \item 130 g Zucker
  \item 130 g Nüsse
  \item 150 g Mehl
  \item 1 verrührtes Eigelb
\end{ingredients}
\begin{directions}
  \item Butter fein schneiden, mit Eigelb, Zucker, Nüssen und Mehl zu einem feinen Teig verarbeiten.
  \item Falls Teig zu weich ist, etwas Mehl zugeben. Teig kaltstellen.
  \item Ca. 1 cm dicke Rollen formen, in 5 cm lange Stücke schneiden.
  \item Mit verrührtem Eigelb bestreichen, auf gefettetem Blech bei schwacher Hitze 10-15 Minuten backen.
\end{directions}
\Notes{Kaltstellen erleichtert das Formen der Rollen.}
\newpage
\section{Butterblumen (Butterplätzchen)}
\index{Butterblumen (Butterplätzchen)}
\RecipeMeta{30}{40}
\begin{ingredients}
  \item 210 g Butter
  \item 2 Eier
  \item 150-160 g Zucker
  \item 1 TL Rum-Aroma
  \item 420 g Mehl
  \item Eigelb
  \item Zucker oder Puderzucker zum Bestreuen
\end{ingredients}
\begin{directions}
  \item Butter, Eier und Zucker zu einer Schaummasse rühren, Rum-Aroma hinzufügen.
  \item Mehl zugeben, mit Knethaken zu einem Teig verarbeiten. Kaltstellen und kurz ruhen lassen.
  \item Teig ca. 0,5 cm dick ausrollen, mit Förmchen ausstechen.
  \item Nach Belieben mit Eigelb bestreichen. Bei mittlerer Hitze 10-15 Minuten backen.
  \item Noch warm in Zucker wenden oder mit Puderzucker bestreuen.
\end{directions}
\Notes{Kaltstellen erleichtert das Ausrollen.}
\newpage
\section{Gefüllte Plätzchen}
\index{Gefüllte Plätzchen}
\RecipeMeta{1}{0}
\begin{ingredients}
  \item 120g Zucker
  \item 150g Mehl
  \item 120g Butter
  \item 3 Eigelb
  \item 150g gemahlene Haselnüsse
  \item Für die Füllung:
  \item 100g gemahlene Haselmüsse
  \item 100g Zucker
  \item 2 EL Rum
  \item 1 Päckchen Vanillezucker
  \item Glasur:
  \item 150g Puferzucker
  \item 1 EL Rum
  \item etwas Wasser
  \item Zum Verzieren:
  \item einige ganze Haselnüsse
\end{ingredients}
\begin{directions}
  \item Mürbteig herstellen
  \item Teig ausrollen
  \item runde Plätzchen ausstechen
  \item bei mittlerer Hitze 15-20 min Backen
  \item Die Zutaten für die Füllung gut vermischen und
  \item zwei Plätzchen damit zusammensitzen.
  \item Mit Guß bestreichen und mit Haselnüssen garnieren.
\end{directions}
\begin{figure}[h]
\centering
\includegraphics[width=0.75\textwidth]{Bilder/Gefüllte Plätzchen.jpg}
\caption{Gefüllte Plätzchen}
\end{figure}
\newpage
\section{Nußkranz}
\index{Nußkranz}
\RecipeMeta{4}{60}
\begin{ingredients}
  \item Etwas Fett
  \item Zutaten für Hefeteig siehe Grundrezept
  \item Zutaten für Füllung:
  \item 1-2 Eier
  \item 100g Zucker
  \item 100-200g Haselnüsse
  \item Saft einer Zitrone
\end{ingredients}
\begin{directions}
  \item Hefeteig nach Grundrezept herstellen
  \item Zutaten der Füllung vermengen
  \item Hefeteig ausrollen und mit der Füllung bestreichen
  \item Hefeteig zu einer Rolle zusammenrollen und
  \item als Ring auf einem eingefetten Backblechen backen.
\end{directions}
\begin{figure}[h]
\centering
\includegraphics[width=0.75\textwidth]{Bilder/Nußkreuz.jpg}
\caption{Nußkranz}
\end{figure}
\newpage
\section{Oma's Quarktorte}
\index{Oma's Quarktorte}
\RecipeMeta{8}{90}
\begin{ingredients}
  \item 125g Butter
  \item 375g Zucker
  \item 1kg Magerquark
  \item 5 gehäufte EL Grieß
  \item 4 Eier
  \item 1 Päckchen Vanillinzucker
  \item 1/2 Päckchen Backpulver
  \item Saft einer Zitrone
\end{ingredients}
\begin{directions}
  \item Alle Zutaten zusammenrühren
  \item Bei 150°C Umluft ca. 1 Stunde backen
\end{directions}
\begin{figure}[h]
\centering
\includegraphics[width=0.75\textwidth]{Bilder/Oma's Quarktorte.jpg}
\caption{Oma's Quarktorte}
\end{figure}
\newpage
\section{Ottilien Kuchen}
\index{Ottilien Kuchen}
\RecipeMeta{8}{60}
\begin{ingredients}
  \item 250g Butter
  \item 200g Zucker
  \item 1 EL Vanilliezucker
  \item 4 Eier
  \item 200g Mehl
  \item 50g Speißestärke
  \item 1 TL Backpulver
  \item 100g Zartbitterschokolade
  \item 50g Zitronat
  \item Zum Verzieren:
  \item 3 EL Konfirüre
  \item 100g Schokokovertüre
  \item 2 EL Mandelblättchen
\end{ingredients}
\begin{directions}
  \item Butter erwärmen, aber nicht schmelzen
  \item Butter, Zucker, Eier schaumig rühren.
  \item Trockene Zutaten separar vermängen und dann in den Teig rühren.
  \item Restliche Zutaten zugeben und unterrühren.
  \item Backmischung in eine gefettete und mit Paniermehl ausgestreutete Form giesen.
  \item Bei 175°C für ca. 45-60min backen.
  \item Warmen Kuchen aus der Form kippen.
  \item Mit Konfitüre bestreichen.
  \item Mit Kovertüre beträufeln.
  \item Mit Mandeln verzieren.
\end{directions}
\begin{figure}[h]
\centering
\includegraphics[width=0.75\textwidth]{Bilder/Ottilien Kuchen.jpg}
\caption{Ottilien Kuchen}
\end{figure}
\newpage
\section{Rotweinkuchen}
\index{Rotweinkuchen}
\RecipeMeta{9}{120}
\begin{ingredients}
  \item 200g Butter
  \item 200g Zucker
  \item 4 Eier
  \item 1 TL Zimt
  \item 1 EL Kakao
  \item 250g Mehl
  \item 1 TL Backpulver
  \item 1/8 Liter Rotwein
  \item 100g Raspelschokolade
\end{ingredients}
\begin{directions}
  \item Butter erwärmen, aber nicht schmelzen
  \item Butter, Zucker, Eier schaumig rühren.
  \item Trockene Zutaten separar vermängen und dann in den Teig rühren.
  \item Restliche Zutaten zugeben und unterrühren.
  \item Backmischung in eine gefettete und mit Paniermehl ausgestreutete Form giesen.
  \item Bei 175°C für ca. 45-60min backen.
  \item Warmen Kuchen aus der Form kippen.
\end{directions}
\begin{figure}[h]
\centering
\includegraphics[width=0.75\textwidth]{Bilder/Rotweinkuchen.jpg}
\caption{Rotweinkuchen}
\end{figure}
\newpage
\section{Schoko-Mandel-Plätzchen}
\index{Schoko-Mandel-Plätzchen}
\RecipeMeta{5}{30}
\begin{ingredients}
  \item 500g Schokolade (Vollmilch oder Zartbitter)
  \item 10-20g Palmin
  \item 175g Cornflakes
  \item 50g Mandeln (blättig gehobelt)
\end{ingredients}
\begin{directions}
  \item Schokolade und Fett im Wasserbad erhitzen
  \item Cornflakes und Mandeln zugeben
  \item Mit einem Teelöffel auf Butterbrotpapier portionieren
  \item Kühlstellen
\end{directions}
\begin{figure}[h]
\centering
\includegraphics[width=0.75\textwidth]{Bilder/Schoko-Mandel-Plätzchen.jpg}
\caption{Schoko-Mandel-Plätzchen}
\end{figure}
\newpage
\section{Waffeln}
\index{Waffeln}
\RecipeMeta{8}{30}
\begin{ingredients}
  \item 200g Butter
  \item 2 EL Honig
  \item 3 Eier
  \item 1 Vanilleschote
  \item 75g Mandeln
  \item 150g Mehl
  \item 1 TL Backpulver
  \item Fett für das Waffeleisen
\end{ingredients}
\begin{directions}
  \item Zutaten vermischen und
  \item Portionsweise im eingefetteten Waffeleisen backen
\end{directions}
\begin{figure}[h]
\centering
\includegraphics[width=0.75\textwidth]{Bilder/Waffeln.jpg}
\caption{Waffeln}
\end{figure}
\newpage
\section{Butterplätzchen}
\index{Butterplätzchen}
\RecipeMeta{1}{0}
\begin{ingredients}
  \item 
\end{ingredients}
\begin{directions}
  \item 
\end{directions}
\begin{figure}[h]
\centering
\includegraphics[width=0.75\textwidth]{Bilder/Butterplätzchen.jpg}
\caption{Butterplätzchen}
\end{figure}
\newpage
\section{Schweizer Fruchtwähen}
\index{Schweizer Fruchtwähen}
\RecipeMeta{8}{30}
\begin{ingredients}
  \item 1/2 Grundrezept Mürbe-, Pasteten-, Hefeteig oder Blätterteig
  \item 30 g gemahlene Mandeln oder Haselnüsse
  \item 500 g Früchte (z. B. Äpfel, Aprikosen, Pflaumen)
  \item 1-2 EL Zucker
  \item 1 Ei
  \item 125 ml Sahne
  \item 1 EL Zucker
  \item Gewürz nach Geschmack (Vanille, Zimt)
\end{ingredients}
\begin{directions}
  \item Teig dünn ausrollen und ein gefettetes rundes Kuchenblech von ca. 30 cm Durchmesser damit auslegen.
  \item Mandeln oder Haselnüsse auf dem Teigboden verteilen.
  \item Früchte vorbereiten und auf dem Teigboden gleichmäßig anordnen.
  \item Mit Zucker bestreuen.
  \item Im vorgeheizten Ofen bei 225–250 °C ca. 15 Minuten backen.
  \item Für den Guss Ei, Sahne, Zucker und Gewürz verquirlen und über die Früchte gießen.
  \item Weitere 15 Minuten backen, bis der Guss fest ist.
  \item Warm oder kalt servieren.
\end{directions}
\Notes{Ergibt ca. 8 Stücke. Backzeit insgesamt ca. 30 Minuten.}
\newpage
\section{Orangenschnitten}
\index{Orangenschnitten}
\RecipeMeta{1}{20}
\begin{ingredients}
  \item 200 g weiche Butter
  \item 150 g Zucker
  \item 1 EL Vanillezucker
  \item 1 Prise Salz
  \item 3 Eier
  \item 250 g Mehl
  \item 2 TL Backpulver
  \item 1 unbehandelte Zitrone
  \item 1 unbehandelte Orange
  \item 200 g Puderzucker
\end{ingredients}
\begin{directions}
  \item Butter schaumig rühren, Zucker, Vanillezucker und Salz hinzufügen.
  \item Eier unterrühren.
  \item Mehl mit Backpulver mischen und unter die Buttermasse geben.
  \item Zitrone waschen, Schale abreiben und Saft auspressen, unter den Teig rühren.
  \item Backofen auf 200 °C vorheizen.
  \item Teig auf ein mit Margarine ausgestrichenes Backblech geben.
  \item Auf der mittleren Schiene ca. 20 Minuten backen.
  \item Orange waschen, Schale abreiben und Saft auspressen.
  \item Saft mit Puderzucker verrühren und den noch warmen Kuchen damit bestreichen.
  \item Orangenschale darüberstreuen.
\end{directions}
\Notes{Backzeit ca. 20 Minuten. Ergibt ein Blech.}
\newpage
\section{Feiner Pfefferkuchen}
\index{Feiner Pfefferkuchen}
\RecipeMeta{1}{20}
\begin{ingredients}
  \item 500 g Honig
  \item 250 g Zucker
  \item 250 g Butter oder Margarine
  \item 100 g Schweinefett
  \item 1000 g Mehl
  \item 200 g gemahlene Mandeln
  \item 200 g gemahlene Haselnüsse
  \item 125 g gehacktes Zitronat
  \item 125 g Kakao
  \item 1 Päckchen Hayma-Lebkuchen-Gewürz
  \item 1½ Päckchen Hayma-Triebkraft
  \item 2 Eigelb
\end{ingredients}
\begin{directions}
  \item Honig, Zucker, Butter oder Margarine und Schweinefett in einem Topf erwärmen und verrühren.
  \item Abkühlen lassen.
  \item Mehl in eine große Schüssel sieben.
  \item Mandeln, Haselnüsse, Zitronat, Kakao, Lebkuchen-Gewürz und Triebkraft dazugeben.
  \item Alles gut vermischen und die abgekühlte Masse sowie Eigelb unterkneten.
  \item Teig kalt stellen.
  \item Teig ½ cm dick ausrollen, Plätzchen ausstechen.
  \item Bei mittlerer Hitze 15–20 Minuten backen.
\end{directions}
\Notes{Klassisches Weihnachtsgebäck. Backzeit: 15–20 Minuten.}
\newpage
\section{Butter-Mandel-Kuchen}
\index{Butter-Mandel-Kuchen}
\RecipeMeta{1}{45}
\begin{ingredients}
  \item 200 g Sahne
  \item 100 g Zucker
  \item 350 g Mehl
  \item 1 Päckchen Backpulver
  \item 1 TL Zimt
  \item 3 Eier
  \item 200 g Butter
  \item 150 g Zucker
  \item 6 EL Sahne
  \item 150 g Mandelblättchen
\end{ingredients}
\begin{directions}
  \item Sahne und Zucker verrühren, Mehl und Zimt zugeben.
  \item Eier unterrühren, Teig auf Blech.
  \item 10 Min. backen, dann Butter-Zucker-Mandel-Mischung aufstreichen.
  \item Weitere 15 Min. backen.
\end{directions}
\Notes{Ergibt ein Blech.}
\newpage
\section{Französischer Orangenkuchen}
\index{Französischer Orangenkuchen}
\RecipeMeta{1}{60}
\begin{ingredients}
  \item 4 Eier
  \item 175 g Zucker
  \item 1 Orange
  \item 75 g Mehl
  \item 75 g Speisestärke
  \item ½ Päckchen Backpulver
  \item 75 g Butter
  \item Saft von 1 Zitrone
  \item 1 EL Zucker
\end{ingredients}
\begin{directions}
  \item Eigelb schaumig rühren, Orangenschale und Zucker zugeben.
  \item Mehl, Stärke, Backpulver mischen, unterrühren.
  \item Eiweiß steif schlagen, unterheben.
  \item Bei 180 °C ca. 50 Min. backen.
  \item Mit Zitronensaft und Zucker tränken.
\end{directions}
\Notes{Ergibt eine Springform.}
\newpage
\section{Windbeutel}
\index{Windbeutel}
\RecipeMeta{10}{45}
\begin{ingredients}
  \item 1/4 l Wasser
  \item 1 Prise Salz
  \item 1 P. Vanillin-Zucker
  \item 60 g Sanella
  \item 125 g Mehl
  \item 1 Ei
  \item 3 Eier
  \item 375 g Schlagsahne
  \item etwas Zucker
  \item Puderzucker
\end{ingredients}
\begin{directions}
  \item Wasser, Salz, Vanillin-Zucker und Sanella aufkochen.
  \item Mehl auf einmal zugeben, abbrennen.
  \item Ein Ei unterrühren, abkühlen lassen.
  \item Restliche Eier unterrühren.
  \item Teighäufchen auf Blech setzen.
  \item Bei 225–250 °C ca. 35 Min. backen.
  \item Deckel abschneiden, auskühlen lassen.
  \item Sahne mit Zucker schlagen, Windbeutel füllen.
  \item Deckel aufsetzen, mit Puderzucker bestäuben.
\end{directions}
\Notes{Ergibt ca. 10 Stück.}
\newpage
\section{Luftige Schnitten}
\index{Luftige Schnitten}
\RecipeMeta{12}{35}
\begin{ingredients}
  \item 1/4 l Wasser
  \item 1 Prise Salz
  \item 1 TL Zucker
  \item 1 P. Vanillin-Zucker
  \item 100 g Sanella
  \item 125 g Mehl
  \item 1 Ei
  \item 3 Eier
  \item 125 g Puderzucker
  \item 2 EL Wasser
  \item 1 EL Rum
  \item gehackte Pistazien
  \item 1/2 l Sahne
  \item 1 EL Zucker
  \item 2 P. Sahnesteif
  \item 1 P. Vanillin-Zucker
\end{ingredients}
\begin{directions}
  \item Wasser, Salz, Zucker, Vanillin-Zucker und Sanella aufkochen.
  \item Mehl auf einmal zugeben, abbrennen.
  \item Ein Ei unterrühren, abkühlen lassen.
  \item Restliche Eier unterrühren.
  \item Teig in Streifen auf gefettetes Blech spritzen.
  \item Bei 225–250 °C ca. 25 Min. backen.
  \item Deckel abschneiden, auskühlen lassen.
  \item Puderzucker mit Wasser und Rum verrühren, Deckel bestreichen, mit Pistazien bestreuen.
  \item Sahne mit Zucker, Sahnesteif und Vanillin-Zucker schlagen.
  \item Schnitten füllen, Deckel aufsetzen.
\end{directions}
\Notes{Ergibt ca. 12 Stück.}
\newpage
\section{Spritzkuchen}
\index{Spritzkuchen}
\RecipeMeta{25}{60}
\begin{ingredients}
  \item 1/2 l Wasser
  \item 125 g Sanella
  \item 1 EL Zucker
  \item 1 Prise Salz
  \item 250 g Mehl
  \item 1 Ei
  \item 5 Eier
  \item Fett zum Ausbacken
  \item Puderzucker
\end{ingredients}
\begin{directions}
  \item Wasser, Sanella, Zucker und Salz aufkochen.
  \item Mehl auf einmal hineinschütten, abbrennen, bis sich die Masse vom Topfboden löst.
  \item Topf vom Herd nehmen, etwas abkühlen lassen.
  \item Ein Ei unterrühren, dann restliche Eier nach und nach unterrühren.
  \item Teig in Spritzbeutel füllen, Ringe auf Backpapier spritzen.
  \item Bei 170–180 °C in heißem Fett 5–6 Min. ausbacken.
  \item Abtropfen lassen, mit Puderzucker bestäuben.
\end{directions}
\Notes{Ergibt ca. 25 Stück.}
\newpage
\section{Marzipanschnitten}
\index{Marzipanschnitten}
\RecipeMeta{12}{50}
\begin{ingredients}
  \item 250 g Mehl
  \item 2 TL Backpulver
  \item 250 g brauner Rohzucker
  \item 1 P. Spekulatiusgewürz
  \item 125 g Mandeln
  \item 125 g Haferflocken
  \item 2 Eier
  \item 250 g Butter
  \item 225 g Erdbeer-Konfitüre
  \item 400 g Marzipanrohmasse
  \item 2-3 EL Milch
\end{ingredients}
\begin{directions}
  \item Teig aus Mehl, Backpulver, Zucker, Gewürz, Mandeln, Haferflocken, Eiern und Butter herstellen.
  \item Teig halbieren, eine Hälfte ausrollen und auf Backblech legen.
  \item Konfitüre und Marzipan darauf verteilen.
  \item Restlichen Teig ausrollen, auflegen, mit Milch bepinseln.
  \item Bei 180°C 35-40 Minuten backen.
\end{directions}
\Notes{Weihnachtlich und aromatisch.}
\newpage
\section{Gewürzschnitten}
\index{Gewürzschnitten}
\RecipeMeta{12}{45}
\begin{ingredients}
  \item 150 g Butter
  \item 150 g Zucker
  \item 4 Eier
  \item 3 TL Pfefferkuchengewürz
  \item 50 g Mondamin
  \item 100 g Mehl
  \item 2 TL Backpulver
  \item 150 g Schokolade
  \item 1 Glas Marmelade
  \item 150 g Puderzucker
  \item 1 EL Rum
  \item 2 EL Wasser
\end{ingredients}
\begin{directions}
  \item Butter und Schokolade schmelzen, mit Zucker, Eiern und Gewürzen verrühren.
  \item Mehl, Mondamin und Backpulver unterrühren.
  \item Teig auf Backblech streichen, bei 175-200°C 25-30 Minuten backen.
  \item Kuchen halbieren, mit Marmelade füllen.
  \item Guss aus Puderzucker, Rum und Wasser herstellen und auftragen.
\end{directions}
\Notes{Würzig und festlich.}
\newpage
\section{Maulwurftorte}
\index{Maulwurftorte}
\RecipeMeta{12}{90}
\begin{ingredients}
  \item 6 Eier
  \item 140 g Butter
  \item 200 g Zucker
  \item 1 Päckchen Vanillinzucker
  \item 100 g Mehl
  \item 1/2 Päckchen Backpulver
  \item 1 Packung Schokoladenraspel
  \item 1 Dose Mandarinen
  \item 2 Bananen
  \item Saft einer Zitrone
  \item 750 g Sahne
  \item 2 Päckchen Sahnesteif
  \item 1 TL Puderzucker
  \item 1 TL Kakao
\end{ingredients}
\begin{directions}
  \item Teig aus Eigelb, Zucker, Butter, Vanillinzucker, Mehl, Backpulver und Schokoladenraspeln herstellen.
  \item Eiweiß steif schlagen und unterheben.
  \item Bei 175°C 45-50 Minuten backen.
  \item Torte aushöhlen, Obst einfüllen, Sahne kuppelartig auftragen.
  \item Mit Kuchenkrümeln, Puderzucker und Kakao bestreuen.
\end{directions}
\Notes{Optisch beeindruckend.}
\newpage
\chapter{Getränke}
\section{Sylvesterpunsch}
\index{Sylvesterpunsch}
\RecipeMeta{1}{30}
\begin{ingredients}
  \item 1 Liter schwarzer Tee
  \item 1 Liter Rotwein
  \item 200-400g Zucker
  \item Schale einer Zitrone
  \item Saft von 2 Zitronen
  \item Satz einer Apfelsine
  \item 1 Stck. Zimtrinde
  \item 6 Gewürznelken
  \item 1/4 Liter Rum
\end{ingredients}
\begin{directions}
  \item Tee zubereiten
  \item Rotwein mit Zucker, Saft und Schale erhitzen bis es anfängt zu schäumen (nicht kochen)
  \item Gewürze, Tee und Rum zugeben.
  \item Eine Weile ziehen lassen
  \item Alles abseihen
  \item noch heiß auftragen
\end{directions}
\begin{figure}[h]
\centering
\includegraphics[width=0.75\textwidth]{Bilder/Sylwesterpunsch.jpg}
\caption{Sylvesterpunsch}
\end{figure}
\newpage
\chapter{Fondue}
\section{Würzmischung für Hackbällchen}
\index{Würzmischung für Hackbällchen}
\RecipeMeta{1}{0}
\begin{ingredients}
  \item Pro Person 250g Hackfleisch (Rind oder Schwein oder gemischt)
  \item 1/2 TL Paprika rosenscharf
  \item je 1 MS Salz und Pfeffer
  \item 1 TL Majoran (getrocknet)
  \item 1 MS Piment
  \item 1 MS Muskat
  \item 1 TL Petersilie (getrocknet)
  \item 1 ML Knoblauch
  \item 1 Ei
  \item Semmelbrösel
\end{ingredients}
\begin{directions}
  \item Zutaten vermischen
  \item mit Teelöffel abstechen und 
  \item mit der Hand rund rollen
  \item Probeklößchen vor dem Servieren kochen und ggf. mit Semmelbrösel strecken
\end{directions}
\begin{figure}[h]
\centering
\includegraphics[width=0.75\textwidth]{Bilder/Würzmischung für Hackbällchen.jpg}
\caption{Würzmischung für Hackbällchen}
\end{figure}
\Notes{z.B. für Brühefondue
Semmelbrösel nach Bedarf, damit Klößchen nicht auseinander fallen}
\newpage
\section{Geflügelfondue mit Ingwersauce}
\index{Geflügelfondue mit Ingwersauce}
\RecipeMeta{4}{45}
\begin{ingredients}
  \item 4 Putenschnitzel (je 100 g)
  \item 4 Hühnerbrustfilets (je 200 g)
  \item 1 frische Ananas
  \item 1 eingelegte Ingwerpflaume
  \item 2 EL Speisestärke
  \item 3 EL Ingwersirup
  \item 1,5 l Geflügelbrühe
  \item 4 EL Sherry
  \item 2 TL Zitronensaft
  \item Salz, weißer Pfeffer
  \item 1 Baguette (400 g)
\end{ingredients}
\begin{directions}
  \item Geflügel in Stücke schneiden, Ananas würfeln.
  \item Ingwersauce aus Ingwerpflaume, Speisestärke, Ingwersirup und Brühe zubereiten.
  \item Mit Sherry, Zitronensaft, Salz und Pfeffer abschmecken.
  \item Brühe erhitzen, Fleisch und Ananas darin garen.
  \item Mit Sauce und Baguette servieren.
\end{directions}
\Notes{Exotisch und leicht.}
\newpage
\section{Hackfleischklößchen-Fondue}
\index{Hackfleischklößchen-Fondue}
\RecipeMeta{4}{60}
\begin{ingredients}
  \item 500 g Hackfleisch
  \item 1 Brötchen
  \item 1 Ei
  \item 1 EL Petersilie
  \item Salz, Pfeffer, Muskat
  \item Majoran oder Oregano
  \item Piment, Paprika
  \item 500 g säuerliche Äpfel
  \item 1 EL Butter
  \item 2 EL Zucker
  \item 1/8 l Wasser
  \item 1/2 Zimtstange
  \item 1 Prise Salz
  \item 2 TL scharfer Senf
  \item 1,5 l Fleischbrühe
\end{ingredients}
\begin{directions}
  \item Brötchen einweichen, mit Hackfleisch, Ei und Gewürzen mischen.
  \item Klößchen formen.
  \item Apfelsauce aus Äpfeln, Butter, Zucker, Wasser, Zimt und Salz kochen, passieren und mit Senf abschmecken.
  \item Brühe erhitzen, Klößchen darin garen.
  \item Mit Apfelsauce servieren.
\end{directions}
\Notes{Preiswert und raffiniert.}
\newpage
\chapter{Süßspeisen}
\section{Apfel-, Kirschen- oder Zwetschgennudeln}
\index{Apfel-, Kirschen- oder Zwetschgennudeln}
\RecipeMeta{15}{60}
\begin{ingredients}
  \item Hefeteig (Grundrezept mittelfest)
  \item 4 Äpfel oder entsprechende Menge Kirschen/Zwetschgen
  \item 500 g Kirschen oder Zwetschgen
  \item 60 g Zucker
  \item 8 g Fett zum Backen
  \item Zucker oder Vanillezucker zum Bestreuen
\end{ingredients}
\begin{directions}
  \item Mittelfesten Hefeteig herstellen und gehen lassen.
  \item Teig in gleichmäßige Stücke teilen, etwas ausrollen und in die Mitte Obst geben.
  \item Teig darüber zusammenschlagen, zu runden Nudeln formen.
  \item In heißem Fett bei Mittelhitze (180–190 °C) backen, bis goldbraun.
  \item Kurz abkühlen lassen, mit Zucker oder Vanillezucker bestreuen und servieren.
\end{directions}
\Notes{Beilage: Kompott oder Vanillesoße.}
\newpage
\section{Rohrnudeln}
\index{Rohrnudeln}
\RecipeMeta{12}{60}
\begin{ingredients}
  \item Mittelfester Hefeteig (Grundrezept Nr. 1211)
  \item ca. 70 g Fett zum Backen
  \item Zucker oder Vanillezucker zum Bestreuen
\end{ingredients}
\begin{directions}
  \item Mittelfesten Hefeteig herstellen und gehen lassen.
  \item Kleine Nudeln abstechen, mit der Hand auf Brett runden.
  \item In zerlassener Butter wenden, nicht zu eng setzen.
  \item Nochmal gehen lassen, dann bei 180–190 °C ca. 30 Min. backen.
  \item Kurz abkühlen lassen, stürzen und sofort mit Zucker oder Vanillezucker bestreuen.
  \item Etwas abgekühlt Nudeln voneinander lösen, auskühlen lassen und servieren.
\end{directions}
\Notes{Beilage: Kompott oder Vanillesoße.}
\newpage
\chapter{Desserts \& Gebäck}
\section{Apfelstrudel aus Nudelteig}
\index{Apfelstrudel aus Nudelteig}
\RecipeMeta{8}{50}
\begin{ingredients}
  \item Nudelteig nach Grundrezept Nr. 917 (300 g Mehl, 1 Ei, Öl)
  \item 3-8 Butterflöckchen
  \item 125 ml saure Sahne
  \item 50 g Rosinen oder Weinbeeren
  \item 100 g Zucker
  \item 50 g Nüsse oder Mandeln
  \item 30 g Butter zum Backen
\end{ingredients}
\begin{directions}
  \item Nudelteig herstellen und in 3 Portionen teilen.
  \item Sehr dünn ausrollen, mit Butterflöckchen und saurer Sahne bestreichen.
  \item Rosinen, Zucker und gehackte Nüsse gleichmäßig darauf verteilen.
  \item Teig locker aufrollen, auf gefettetes Backblech legen.
  \item Mit Butter bestreichen und bei 175 °C Ober-/Unterhitze 30–40 Min. backen.
  \item Kurz ruhen lassen, in Portionen schneiden und mit Puderzucker bestreut servieren.
\end{directions}
\Notes{Handschriftliche Notiz: '150 Umluft, 175 Ober-/Unterhitze'. Strudel hat etwas blättrige Beschaffenheit.}
\newpage
\newpage
\printindex
\newpage
\listoffigures
\end{document}
